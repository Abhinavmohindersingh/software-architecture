\documentclass{csse4400}

\usepackage{tikz}
\usetikzlibrary{positioning}
\usetikzlibrary{arrows}

\usepackage{float}

\usepackage{enumitem}

\usepackage{languages}

\title{DevOps}
\author{Richard Thomas}

\date{\week{4}}
\begin{document}

\maketitle

\section{Brief}
DevOps is a \link{portmanteau}{https://www.britannica.com/topic/portmanteau-word} of development and operations.
It is intentionally a portmanteau to emphasise that the approach requires a close integration
of development and operations behaviours in one team.
In a proper DevOps environment the development team is responsible for the infrastructure on which their software runs,
not just for the software itself.
This can include considering infrastructure costs in their development budget and estimates.

\noindent
This week we will
\begin{enumerate}
    \item provide an overview of DevOps,
    \item consider what are necessary DevOps practices, and
    \item let you explore how you might implement a DevOps pipeline.
\end{enumerate}


\section{Introduction to DevOps}
You should be familiar with the concepts of automated testing and continuous integration.
We will now extend that to include continuous testing and deployment,
and then expand on these to provide a full DevOps process.

For an introduction to DevOps read Amazon's description of \link{``What is DevOps?''}
{https://aws.amazon.com/devops/what-is-devops/} \cite{AWS-DevOps}.
Do not worry about the discussion about microservices.
That is an architectural style that will be covered later in the course.
DevOps does not require a microservices architecture,
though there are some benefits of using it.

Skim through the description of implementing DevOps at Wotif\footnote{Now Expedia.} in
\link{``DevOps: Making it Easy to Do the Right Thing''}
{https://search.library.uq.edu.au/permalink/f/tbms52/TN_cdi_webofscience_primary_000383092600012CitationCount}
\cite{CallananMatt2016DMIE}.
Focus on the section titled ``Making it Easy.''
This is where they discuss how they provided the environment to support DevOps.


\section{DevOps Practices and Tools}\label{sec:DevOps-Practices}
You should be familiar with some of the practices and tools used in a DevOps process.
Amazon's description of \link{``What is DevOps?''}
{https://aws.amazon.com/devops/what-is-devops/} \cite{AWS-DevOps} listed what they consider to be required practices.
For any practices (aside from microservices) that you are not familiar with, you should follow the links to read a summary of those practices.
Amazon naturally describes their tools for implementing a DevOps pipeline.
You should skim the description of Amazon's tools at \link{``DevOps and AWS''}
{https://aws.amazon.com/devops/} \cite{AWS-DevOps-Tools}.
You do not need to be familiar with these tools, but should have a general idea of what services the tools provide from their summaries.

\newpage
\noindent
Another view of necessary DevOps practices is that it requires continuous
\begin{itemize}[nosep]
    \item development,
    \item integration,
    \item testing,
    \item operations,
    \item deployment,
    \item monitoring, and
    \item feedback.
\end{itemize}
These can only be achieved through automation.

\subsection{Pre-Tutorial Task}
\textit{\textbf{Before}} the tutorial, identify a tool that can be used for each of the seven practices listed above.
Come to the tutorial with a list of tools and be prepared to give a ten second summary of each tool.

\subsection{Discussion Question}
In the tutorial you will describe the tools you identified. Briefly explain how they can be integrated to deliver a complete DevOps pipeline.

\teacher{This could be done by having students go around their table and describe the tools they identified to everyone else at their table.
Assuming a group of students haven't come prepared with a list of tools,
challenge them to spend 5 minutes finding tools possible tools.}


%\section{Best Practices}
% Possibly add some commentary about best practices.


\section{DevOps in Practice}
At least skim read \link{``DevOps Capabilities, Practices, and Challenges: Insights from a Case Study''}
{https://search.library.uq.edu.au/permalink/f/tbms52/TN_cdi_arxiv_primary_1907_10201}
\textbf{\textit{before}} the tutorial \cite{SenapathiMali2018DCPa}.

\subsection{Discussion Question}
At least skim read the article \textit{\textbf{before}} the tutorial.
In the tutorial you will discuss the differences between the capabilities and technological enablers mentioned in the article,
and the seven DevOps practices listed in section \ref{sec:DevOps-Practices}.
\begin{itemize}
    \item Do the seven necessary DevOps practices map perfectly to the enablers in the article by Senapathi et al \cite{SenapathiMali2018DCPa}?
    \item If there are some mismatches, discuss whether they are important enough that they should be a required practice.
\end{itemize}


\section{DevOps Pipeline}
For the service-based architecture approach of the Sahara eCommerce case study \cite{service-based-slides},
describe a DevOps pipeline that could be employed for the project.
\begin{enumerate}
    \item What types of tools would be required?
    \item Which specific tools would you choose?
    \item On which type of computing infrastructure would you deliver the system?
    \item What parts of the deployment and operations processes could be automated?
\end{enumerate}

\noindent
You will work in small groups to identify a set of tools that can be used to create a DevOps pipeline.

\teacher{Get students to work on the pipeline with everyone else at their table.
If there is time, get them to give a 2 minute overview of their pipeline to the rest of the class.}


\bibliographystyle{ieeetr}
\bibliography{articles,ours}

\end{document}