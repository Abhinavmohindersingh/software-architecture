\documentclass{csse4400}

%\teachermodetrue

\usepackage{tikz}
\usetikzlibrary{positioning}
\usetikzlibrary{arrows}

\usepackage{float}

\usepackage{enumitem}

\usepackage{languages}

\title{Chatterbox}
\author{Brae Webb}

\date{\week[tutorial]{10}}
\begin{document}

\maketitle

\section{Brief}

Text-to-speech software supports accessibility,
enables smart-home devices,
and even breaks down language barriers.
Unfortunately, text-to-speech is computationally intensive.
While the technology has made great advances over the past few decades,
many open-source implementations are still inefficient.

The University of Queensland is looking to convert all course content into speech.
This will support visually impaired students in their studies.
All course content from slack messages and blackboard announcements to textbooks must be converted to speech.
You are responsible for designing and implementing a service to generate synthesized speech for use across the entire university.


\section{Requirements}

As you might imagine,
blackboard announcements occur frequently and should be translated in almost real time.
While textbooks are set ahead of semester and may take several days to process.
The university will experience peaks of usage.
At the start of semester,
instructors set many textbooks which need to be processed.
The university will also experience usage lows over the summer holiday period when few translations will be required.
The university is not willing to pay for the resources required during usage peaks all year round.
Your implementation must be able to scale dynamically based on the current amount of jobs to be processed.
Your implementation must use AWS cloud resources.

Your implementation must implement the following API specification:\\
\url{https://csse6400.uqcloud.net/api/chatterbox}

\section{Outline}

\subsection*{Introduction (5 minutes)}
Introduction to the brief
and resources including the API specification and quality scenarios.

\subsection*{Planning (10 minutes)}
In small groups, begin a discussion,
the discussion may include:

\begin{enumerate}
    \item What are the key requirements introduced by the quality scenarios?
    \item What strategies can you implement to support these scenarios?
    \item What AWS resources would prove helpful?
    \item What are the likely bottlenecks?
\end{enumerate}

\subsection*{Design (20 minutes)}

In your group, design an appropriate system.
You need to consider the flow of an API request through your system,
use the API specification to ensure all use cases have been considered.

\subsection*{Presentation (15 minutes)}

In the remaining time,
each group should present their proposed design.
This is an opportunity for discussion amoungst the class to point out limitations of the proposed system designs.

\section{Quality Scenarios}\label{sec:scenarios}

\paragraph{Semester break}
Your application will see decreased usage during the semester break.
Minimal students and staff may periodically query pre-generated speech when revising content.
During this period we might see traffic of not more than 20 parallel users,
querying read-only endpoints.

\paragraph{Exam revision}
During the 20 minutes prior to an exam of a large course, such as CSSE1001,
you might expect the entire cohort of over one thousand students to be revising the content.
Revising the content will consist primarily of read-only requests.

\paragraph{Monday of exam block}
Courses with large enrolments tend to have their exam scheduled at the start of exam block.
This gives course staff the most possible time to mark these exam papers.
At 8am on Monday of exam block,
the 4 largest cohorts will begin using your service for revision.

In addition, teaching staff for exams later in the week have begun uploading revision material.
We might expect about 30 staff members to upload their one page course summaries,
about 3,500 characters each.

\paragraph{Teaching cancelled}
Queensland has seen `unprecedented' rainfall causing the university to flood.
The university admin have made the call to cancel teaching for the week.
Coordinators are responsible for disseminating this information to their students.
All 1,957 courses must send announcements averaging 100 characters (many exact copies of the suggested announcement),
to relay this information.

\paragraph{Reading list due date}
The library has a deadline for course coordinators to submit their course reading lists.
In this completely fictional example,
we can imagine that coordinators have left uploading their reading list until the last minute.
Suddenly your system receives a large amount of requests to generate the spoken version of textbooks (we will vastly under-estimate that textbooks are around 7,000 characters).
About 100 coordinators submit their reading list at the last minute.

\paragraph{In-semester Monday morning}
Each semester there are a small number of new courses running for the first time.
These courses, in an attempt to keep up with the passage of time,
upload the course material for the week on a Monday morning.
About 20 courses uploading 2,500 characters of material.
At the same time, many coordinators like to send out announcements on Monday mornings summarising the week's content.
These announcements are typically around 80 characters long and sent out by 75 or so coordinators.

\end{document}
