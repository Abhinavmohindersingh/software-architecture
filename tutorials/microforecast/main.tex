\documentclass{csse4400}

\usepackage{tikz}
\usetikzlibrary{positioning}
\usetikzlibrary{arrows}

\usepackage{float}

\usepackage{enumitem}

\usepackage{languages}

\title{MicroForecast}
\author{Brae Webb}

\date{\week[tutorial]{3}}
\begin{document}

\maketitle

\section{Brief}

You are tasked to design Telstra's new hyper-local weather application, MicroForecast.%
\footnote{\url{https://exchange.telstra.com.au/were-testing-a-hyper-local-weather-network-for-australias-farmers/}}
The hyper-local weather application uses weather information from various sources in Telstra's IoT mobile network,
including mobile phones.
The application will provide users with up-to-date information about the weather conditions in their immediate vicinity.

\section{Requirements}

\begin{enumerate}
    \item The team is currently planning to collect and process weather data from:
        \begin{itemize}
            \item street cameras;
            \item cell phone towers;
            \item drones;
            \item airplanes; and
            \item mobile phones.
        \end{itemize}
        Your design must be able to support these interfaces and be able to easily extend to support new sources of data.
    \item Metrics including temperature, humidity, wind speed, and precipitation probability should be supported as data sources.
        However, your design should be designed to be easy to extend to support more data sources in the future.
    \item The application should display current weather conditions for the user's current location.
    %\item The application should allow users to view a 5-day weather forecast for their current location.
    \item Provide weather alerts for severe weather conditions such as thunderstorms or flooding.
    %\item The application should be fast and responsive, even when processing large amounts of data.
    \item The application relies on users sharing their data. It is a high priority that this data remains secure and protects user privacy.
\end{enumerate}

\section{Outline}

\subsection*{Introduction (5 minutes)}
Introduction to the brief and hyper-local weather applications.

\subsection*{Design (10 minutes)}
In teams, discuss and sketch out a potential design for the system.
You can use any tools you like, but you should be able to explain your design to the class.
If you are using digital tools, \link{excalidraw}{https://excalidraw.com/} is useful for sketching.
Your design does not need to be complete nor perfect,
try to be creative so that we can discuss the pros and cons of various design options.

\subsection*{Discussion (10 minutes)}
With the class, present a few of the designs and discuss the pros and cons of each.
Consider the following questions:
\begin{itemize}
\item Which quality attributes are prioritized in this design?
\item How would you extend this design to support more weather data sources?
\item How would you extend this design to support more types of data?
\item How could updates to the data sources be handled?
\item Is user data treated appropriately?
\item Are there trade-offs in this design?
\end{itemize}

\subsection*{Sketching (20 minutes)}
Individually sketch out a basic implementation of your preferred design from the discussion.
Your design sketch should include:
\begin{itemize}
\item A high-level overview of the architecture, including any architectural patterns used.
\item A description of the data sources and the format of the data they provide.
\item An sketch of the required interfaces that data source devices need to implement.
\item A free-form diagram that illustrates the communication between components of the system.
\end{itemize}
You may include any other details including pseudocode, class diagrams, etc. that you think are relevant.

\subsection*{Optimization (10 minutes)}
Discuss how you would optimize your design to improve extensibility, performance, scalability, etc. in a real system.
Consider the following questions:
\begin{itemize}
\item What is the burden to implement a new data source?
\item Telstra's IoT network is designed for millions of devices, how would this system scale if it was deployed across the network?
\item Are there any security concerns?
\end{itemize}

\section{Design Challenges}

\subsection*{Challenge 1: Data Consistency}
Different weather data sources may provide conflicting data.
For example, one source may report that it is currently raining while another source reports that it is not.
How would you modify your design to ensure data consistency across multiple sources?

\subsection*{Challenge 2: Mobile Data Usage}
Users have limited data on their mobile plans and would like to minimize data usage.
How would you design the application to minimize data usage while still providing the desired functionality?

\subsection*{Challenge 3: Outdated Weather Information}
Weather data is constantly changing and can become outdated quickly.
How would you ensure that users receive up-to-date weather information?

\end{document}
