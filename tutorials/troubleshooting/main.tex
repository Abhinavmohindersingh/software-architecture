\documentclass{csse4400}

\usepackage{languages}

\title{Troubleshooting the Web}
\author{Larene Le Gassick \& Brae Webb}

\date{\week{5}}
\begin{document}

\maketitle

\section{Brief}

This week we will gain hands-on experience with using the browser development console introduced in the lecture.
Specifically, we will investigate the internals of the Rolex Sydney Hobart Yacht Race:

\noindent\url{https://www.rolexsydneyhobart.com}

The rest of the tutorial sheet outlines some activities for investigating the website.
Use the final page of the document to record any of your findings while completing the activities.
Once you have a reasonable understanding of the webpage,
scribble a container diagram you imagine reflects the implementation.

\section{Activity: Exploring Endpoints}
Most webpages use API requests for data to populate the contents of a webpage.
Explore the website and its features.
Use the developer console to identify the various API end-points,
include information such as the URL, HTTP method, request headers, request data, etc.
You can write these end-points down or add them to Postman.
Also note any pages you find which do not use API requests to populate the content.

\section{Activity: Explore Tooling}
Explore the folder structure of the website.
Try and use this folder structure to identify the key frameworks and libraries that are used.
If you are able,
use other developer console features to try and identify what backend frameworks might be utilized.

\section{Activity: Persistent Data}
In the Yacht Tracker,
you can favourite yachts and only show them on the map.
Once you favourite a yacht,
it stays favourited when you refresh or close the browser and open it again.\vspace{-0.6em}
\begin{itemize}
    \setlength\itemsep{-0.5em}
    \item Where is this favourite data stored?
    \item How do you remove the favourites using the browser developer tools?
\end{itemize}

\section{Activity: AWS Services}
The Rolex Sydney to Hobart website reaches a load of a peak of 15,000 international viewers concurrently during the beginning and end of the race.
They are mostly refreshing the tracking screen.
What kind of AWS infrastructure would you recommend to handle that load that is also budget friendly?

\section{Bonus Activity: Bug Fixing}
Changes are not showing up in production that you just deployed,
and you made changes to some styles in \texttt{rshyr-rolex.css} 
--- what would you need to do to get the changes to show up?

\clearpage

\newcommand{\fillbox}[1]{%
\noindent\fbox{
\begin{minipage}[t][0.3\textheight][t]{0.45\textwidth}
#1
\end{minipage}
}}

\fillbox{\small API end-points}
\fillbox{\small Pages not using API calls}

\fillbox{\small Frameworks used}
\fillbox{\small Libraries used}

\fillbox{\small Ideas for improvement}
\fillbox{\small Notes \& scribbles}
% \fillbox{Hello}
% \fillbox{Blah}

\noindent\fbox{
\begin{minipage}[t][\textheight][t]{\textwidth}
Draw an imagined container diagram here:
\end{minipage}
}

% \fillbox{Hello}
% \fillbox{Blah}


% \bibliographystyle{ieeetr}
% \bibliography{articles}

\end{document}