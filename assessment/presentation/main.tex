\documentclass{csse4400}

\usepackage{CJKutf8}
\usepackage{multirow}
\usepackage{xltabular}
\usepackage{pdflscape}

\title{Case Study Presentation}
\author{Richard Thomas \& Brae Webb}
\date{Semester 1, 2022}

\begin{document}
\maketitle

\section*{Summary}
In this assignment, you will be asked to demonstrate your ability to
\textsl{understand}, \textsl{communicate}, and \textsl{critique} an architecture of an existing software project.
\begin{enumerate}
    \item You will present the key information about the architecture of the project you documented in your case study assignment to your practical class.
    \item This will include an updated critique of the architecture and any other relevant updates to the information originally provided in your report.
\end{enumerate}


\section{Introduction}
You will give a presentation describing the project you selected for the documentation assignment.
The intent is to give everyone in the course a broader view of how software architectures are used to solve problems.
Your presentation should take advantage of what you have learnt in this course since you submitted your documentation report.
This may allow you to provide a more insightful critique of the architecture or to provide a more accurate description of the project's architecture.


\section{Presentation Content}
You are free to structure your presentation however you wish, though you should use some form of slides to support the delivery of information.
Your presentation needs to deliver the following content.

\begin{description}
    \item[Title Slide] Name of the software project, and your name and student number.
    \item[Introduction] Describe the software project, explaining the its key functionality and target users.
    \item[Quality Attributes] Describe the quality attributes of most importance to the project.
    \item[Context] Provide an overview of the software system's context and its external dependencies.
    \item[Architecture] Describe the software's architecture.
    \item[Critique] Analyse the software's architecture, describing how well it delivers its ASRs\footnote{Architecturally Significant Requirements}.
    \item[Conclusion] Highlight the key points or lessons learnt about the software's architecture.
\end{description}

Your presentation should introduce the software project.
Give an elevator pitch style summary of what problem the project solves and its key features.
Describe which quality attributes you think are most important for the project, and why.
Describe the project's software architecture. Use appropriate views and notation to convey the important aspects of its architecture.
Summarise your critique of the software architecture, highlighting how well it supports delivering the project's architecturally significant requirements.

Your audience is other students in this course. You may assume the audience has knowledge of the course content,
though you should not assume they are familiar with the project you are describing.


\section{Presentation}
Presentations will take place in your practical class sessions during weeks 10 to 13.
You will have a \textbf{maximum} of eight minutes for your presentation, plus three minutes for questions.
There is no minimum time required for your presentation,
it is up to you to determine when you have described all relevant information about the software architecture within your eight minute limit.

If your presentation exceeds eight minutes, the marker will ask you to stop your presentation.
No content of your presentation past eight minutes will be marked.


\section{Identity Verification}
The presentation is an identity verified assignment.
You must make your presentation in-person.
The marked result of your presentation will be used to determine any caps applied to your grade.
(That means a late penalty on the submission of your slides will not affect the mark used to determine a grade cap.)
The first slide of your presentation \textbf{must} contain your full name, as recorded in UQ's student enrolment system, and student number.

\subsection{Online Identity Verification}
If you are presenting online, at the start of your presentation you \textbf{must} show your UQ student card (it does not matter if your UQ student card has expired),
or official government photo id that shows your full name.
Your id must be clearly visible for at least 3 seconds.
If a marker cannot view your card clearly enough, they will ask you to move it so it is clearly readable.

\begin{CJK*}{UTF8}{gbsn}
If your government id does not show your name in Roman characters,
as recorded in UQ's student enrolment system, you need to include a clear image of your government id on your first slide and a textual
representation of your name that can be selected and copied from your slide so that it may be pasted into a translator.
(e.g. If you use your China Resident Identity Card, you must provide clear images of the front and back
of the card. You also need to provide a textual representation of your name in Chinese characters, e.g. 蒙晶.)
\end{CJK*}

Your face must be visible throughout the presentation to show that you are the one speaking during the presentation.
This may be through Zoom's participants window.
If you cannot arrange for your face to be visible throughout the presentation,
you \textbf{must} contact the course coordinator before 3 May 2022 to discuss your constraints.

\subsection{On-Campus Indentity Verification}
If you are presenting on-campus, at the start of the practical session in which you are presenting
you \textbf{must} show the marker your valid UQ student card.
Like in an exam situation, if you have lost your student card
you must obtain a temporary identity verification document from the UQ student centre \emph{before} the presentation.


\section{Submission}
There are three components that make up your assessable content for the presentation.
These are the slides you use for your presentation, the presentation itself, and your evaluation of other presentations.

\subsection{Slides}
The slides for your presentation are to be submitted to a link provided on BlackBoard.
Your slides are due at \textbf{13:00 (AEST) on 3 May 2022}.
Late submission of your slides will result in a penalty of 2\% of the maximum possible marks for the presentation, per minute that they are late.
Regardless of any penalty applied to the presentation, \emph{even} if the penalty is 100\%,
you \textbf{must} still make your presentation in your allocated timeslot.

\subsection{Presentation}
The presentations will take place in the practical sessions during weeks 10 to 13.
You will be allocated a week in which you are to make your presentation.
Your presentation is to use the slides you submitted on the third of May.

If you do not deliver your presentation, your final grade will be capped at a failing grade.
If you are unable to attend your session to give your presentation due to exceptional circumstances,
you may apply to defer your presentation to another date.
You are not able to defer a deferred presentation.

\subsection{Peer Assessment}
You are expected to attend all presentations.
You are required to submit an evaluation of each presentation you observe.
Submission of \emph{meaningful} feedback for at least \textbf{90\%} of the presentations in your practical class
is required to obtain a mark of 40\% or higher in the presentation assessment.

An online form will be provided for you to submit your assessment for each presentation.
You must submit your assessment of each presentation separately in order for the system to record all of your assessments.

If you are unable to attend a practical session due to exceptional circumstances,
and miss viewing several presentations,
you may apply for a modified limit on the number of presentations you must assess.

\section{Academic Integrity}
As this is a higher-level course, you are expected to be familiar with the importance of academic integrity in general, and the details of UQ's rules.
If you need a reminder, review the \link{Academic Integrity Modules}{https://web.library.uq.edu.au/library-services/it/learnuq-blackboard-help/academic-integrity-modules}.
Submissions will be checked to ensure that the work submitted is not plagiarised.
If you have quoted or paraphrased any material from another source, it must be correctly \link{cited and referenced}{https://web.library.uq.edu.au/node/4221/2}.
Use the \link{IEEE referencing style}{https://libraryguides.vu.edu.au/ieeereferencing/gettingstarted} for citations and your bibliography.

Uncited or unreferenced material will be treated as not being your own work.
Extensive quotation or minor rephrasing of material from cited sources should be avoided.
Significant amounts of cited material from other sources, even if paraphrased, will be considered to be of no academic merit.
In all cases, any material that you cite must support the arguments and points that you are making in your presentation.


\clearpage
\begin{landscape}

\section{Criteria}

\fontsize{9}{11}\selectfont

\begin{xltabular}{\linewidth}{| p{2cm} | X | X | X | X | X |}
\hline
\multicolumn{1}{|c}{\multirow{2}{*}{\textbf{Criteria}}} &
  \multicolumn{5}{c|}{\textbf{Standard}} \\ \cline{2-6} 
\multicolumn{1}{|c}{} &
  \multicolumn{1}{c|}{\textbf{Advanced (20)}} &
  \multicolumn{1}{c|}{\textbf{Proficient (16)}} &
  \multicolumn{1}{c|}{\textbf{Developing (13)}} &
  \multicolumn{1}{c|}{\textbf{Emerging (9)}} &
  \multicolumn{1}{c|}{\textbf{No Evidence (0)}} \\ \hline
\endhead
%
\textbf{Content\newline 75\%} &
Project was introduced clearly and well situated within its context, establishing a good framework for the rest of the presentation.

ASRs were clearly described, well justified, and are clearly of high importance for the project.

Architecture description was clear, complete, concise, informative and at an appropriate level of detail.

Critique was clear, insightful, concise and demonstrates an in-depth knowledge of the system design. &
Full system functionality is well defined and appears to describe a complete system.\newline\newline\newline MVP is fairly well defined, appears to be minimal, and seems feasible. &
System functionality is fairly clear but appears to be missing one or two aspects of the system.\newline\newline MVP is generally clear but lacks some important aspects or contains too many features, but still may be feasible. &
System functionality is not very clear or is missing a few aspects of the system.\newline\newline\newline MVP lacks some important information, or is too small or large, or is not feasible. &
System functionality is vague or contradictory, or it is missing several aspects of the system.\newline\newline\newline MVP lacks important information, or is far too small or large, or is not feasible. \\
\hline
\textbf{Organisation\newline 10\%} &
Information was logically sequenced, with clear objectives and signposting to make it easy to follow.

References were used well, throughout presentation, to support points being made and were always cited correctly. &
All quality attributes are important, their selection is fairly well justified, and there are no other obviously more important attributes.\newline\newline\newline They seem to be measurable or testable. &
All quality attributes seem important, their selection is mostly adequately justified, and any other potential important attributes are not too much more important.\newline\newline Most seem to be measurable or testable. &
Some quality attributes are important, their selection is weakly justified, and there appear to be other more important attributes.\newline\newline\newline\newline Most are not described in a way to indicate how they can be measured or tested. &
Quality attributes are not important, or their selection is poorly justified, or there are clearly more important attributes.\newline\newline\newline\newline Their descriptions make it difficult to see how they can be measured or tested. \\
\hline
\textbf{Presentation\newline 15\%} &
Presenter spoke fluently, clearly and audibly.

Language was always pitched at a good technical level for audience.

Entire presentation was very well paced and appeared well practised.

Visual aids were informative and effective, and not distracting. &
Evaluation plan is fairly clearly described and seems to be mostly feasible.\newline\newline It covers almost all functionality of the MVP and all quality attributes. &
Evaluation plan is comprehensible and does not appear to be too difficult to implement.\newline\newline It covers most functionality of the MVP and most quality attributes. &
Evaluation plan is not clear or does not appear to be feasible.\newline\newline\newline It covers some functionality of the MVP and at least the most important quality attribute. &
Evaluation plan is confusing or contradictory or is clearly not feasible.\newline\newline It covers little functionality of the MVP or, at best, less important quality attributes. \\
\hline
%\textbf{Documentation\newline 10\%} &
%The document structure leads the reader to a clear understanding of the proposal.\newline\newline\newline It is at an appropriate technical level.\newline\newline Grammar and prose enhance the clarity of the document. &
%The document is logically structured.\newline\newline\newline\newline It is at an appropriate technical level.\newline\newline Grammar and prose are appropriate for a professional document. &
%The document structure does not hinder comprehension.\newline\newline\newline\newline It is mostly at an appropriate technical level.\newline\newline Grammar and prose do not hinder comprehension. &
%The document is not logically structured.\newline\newline\newline\newline It is mostly at an appropriate technical level.\newline\newline Grammar and prose hinder comprehension a little. &
%The document is poorly structured, requiring the reader to reference other sections to understand the content.\newline\newline It is not at an appropriate technical level.\newline\newline Grammar and prose make comprehension difficult. \\
\hline
\end{xltabular}

\end{landscape}

%\bibliographystyle{ieeetr}
%\bibliography{articles}

\end{document}