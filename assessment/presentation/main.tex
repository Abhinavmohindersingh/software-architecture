\documentclass{csse4400}

\usepackage{CJKutf8}
\usepackage{multirow}
\usepackage{array}
\usepackage{xltabular}
\usepackage{pdflscape}
\usepackage{enumitem}

\newcolumntype{P}[1]{>{\centering\arraybackslash}p{#1}}

\title{Case Study Presentation}
\author{Richard Thomas \& Brae Webb}
\date{Semester 1, 2023}

\begin{document}
\maketitle

\section*{Summary}
In this assignment, you will be asked to demonstrate your ability to
\textsl{understand}, \textsl{communicate}, and \textsl{critique} an architecture of an existing software project.
You will
\begin{enumerate}
    \item choose a suitable open source software project that has non-trivial functionality and architecture;
    \item present the key information about the architecture of your selected project; and
    \item critique the architecture.
\end{enumerate}


\section{Introduction}
The digital world relies heavily on open source software, as seen by the \link{log4j vulnerability}
{https://www.cisa.gov/news-events/news/apache-log4j-vulnerability-guidance}.
Fortunately, open source developers often maintain high quality documentation for the users of their projects.
Unfortunately however, many open source projects do not maintain the same high quality documentation
for the architecture of their software projects. This can cause difficulty for developers who want to
contribute to the project, but first need to understand it.

In your presentation, you have the chance to right this wrong.
You are to find an open source software project with a sufficiently complex architecture and describe it.
You may choose to share your results with the project developers.
You are encouraged to do this, as the perspective of a newcomer to a project is often invaluable to the seasoned developers.

Before looking for projects, read some of the 
\link{architecture documentation}{https://delftswa.gitbooks.io/desosa2016/content/} written by students at TU Delft.
You may also find it useful to read through one or more of the architecture descriptions in either volume of
\link{\textit{The Architecture of Open Source Applications}}{https://aosabook.org/en/index.html}.

You will give a presentation describing the architecture of the project you select.
The intent is to give everyone in the course a broader view of how software architectures are used to solve problems.
Your presentation should take advantage of what you learn throughout this course.
You are to critique the architecture, discussing how well it meets the projects goals.


\section{Finding a Project}
Criteria for selecting an open source software project:
\begin{itemize}[topsep=4pt,partopsep=1pt,itemsep=2pt,parsep=2pt]
    \item The project cannot be covered in the tutorials, by the TU Delft students in the link above,
          nor in \textit{The Architecture of Open Source Applications}.
    \item The project must have at least one release within the last year.
    \item The project must be an executable system or library with a significant software structure.
          (e.g. A CSS library is \textit{not} executable \textit{nor} does it have a significant software structure.)
\end{itemize}

\noindent%\begin{minipage}{\textwidth}
 Places to look for projects:
\begin{itemize}[topsep=4pt,partopsep=1pt,itemsep=2pt,parsep=2pt]
    \item GitHub explore page: \url{https://github.com/explore};
    \item Apache project list: \url{https://apache.org/index.html#projects-list};
    \item Awesome Open Source \url{https://awesomeopensource.com/};
    \item in class discussion with other students;
    \item or, ask your tutor.
\end{itemize}
%\end{minipage}


\section{Presentation Content}
You are free to structure your presentation however you wish,
though you should use some form of slides to support the delivery of information.
Your presentation needs to deliver the following content.

\begin{description}
    \item[Title Slide] Name of the software project, and your name and student number.
    \item[Introduction] Describe the software project, explaining the its key functionality and target users.
    \item[ASRs] Describe the Architecturally Significant Requirements (ASR) of most importance to the project.
    \item[Context] Provide an overview of the software system's context and its external dependencies.
    \item[Architecture] Describe the software's architecture.
    \item[Critique] Analyse the software's architecture, describing how well it delivers its ASRs.
    \item[Conclusion] Highlight the key points or lessons learnt about the software's architecture.
\end{description}

Your presentation should introduce the software project.
Give an elevator pitch style summary of what problem the project solves and its key features.
Describe which ASRs and, in particular, the quality attributes you think are most important for the project, and why.
Describe the project's software architecture using appropriate views \cite{view-notes}.
Critique the software architecture, highlighting how well it supports delivering the project's architecturally significant requirements.

You should describe any security risks inherent in the software architecture.
Your critique should evaluate and discuss what security design principles appear to have been followed in the design of the software and how well they guard against the security risks.

Your description of the software architecture should cover all of its important aspects.
You are not expected to get down to the level of describing the detailed design of the software.
You should not need to provide class or dynamic diagrams for the entire system.
You may need to provide a small number of class or dynamic diagrams to highlight important features supported by the architecture.
For example, a class diagram showing a plug-in API and a dynamic diagram of how the application uses a plug-in, may be informative.

Your audience is other students in this course. You may assume the audience has knowledge of the course content,
though you should not assume they are familiar with the project you are describing.

\subsection{Citations \& References}
You may use references in your presentation to support points you are making.
These must be cited and referenced using the \link{IEEE referencing style}
{https://libraryguides.vu.edu.au/ieeereferencing/gettingstarted}.
The final slide(s) of your presentation should include the references to any cited material.
You should display the reference slide(s) for about 3 seconds at the end of your presentation.
You are not required to speak to the reference slides,
aside from possibly thanking your audience for listening and stating these are your references.


\section{Presentation}
Presentations will take place in your practical class sessions during weeks 10 to 13.
You will have a \textbf{maximum} of eight minutes for your presentation, plus three minutes for questions.
There is no minimum time required for your presentation,
it is up to you to determine when you have described all relevant information about the software architecture within your eight minute limit.

If your presentation exceeds eight minutes, the marker will ask you to stop your presentation.
No content of your presentation past eight minutes will be marked.

As a presenter, you should not read a script.
You may wish to write a script to prepare for the presentation but should not read it during the presentation.
You may make use of notes during the presentation but you should only quickly glance at your notes to keep yourself on track.
You should not be constantly referring to notes.
You should try to maintain eye contact with your audience, rather than focussing on your notes or slides.

A schedule of presentation time slots will be organised after the mid-semester break.
You will be allocated a time to present in one of the practical class sessions.
Please inform the course coordinator of any constraints you may have regarding presentation times before Easter.


\section{Identity Verification}
The presentation is an identity verified assignment.
You must make your presentation in-person.
At the start of your presentation you must show your UQ student card to one of the markers at your session.

The marked result of your presentation will be used to determine any caps applied to your grade.
(That means a reduction in grade level because you did not pass the draft architectural model demonstration
in the week 6 tutorial will \textbf{\textit{not}} affect the mark used to determine a grade cap.)
The first slide of your presentation \textbf{must} contain your full name, as recorded in UQ's student enrolment system, and student number.

\subsection{On-line Identity Verification}
If you are are an external student it does not matter if your UQ student card has expired.
If you do not have a UQ student card, you may use an official government photo id that shows your full name.
Your id must be clearly visible for at least 3 seconds.
If a marker cannot view your card clearly enough, they will ask you to move it so it is clearly readable.

\begin{CJK*}{UTF8}{gbsn}
If your government id does not show your name in Roman characters, as recorded in UQ's student enrolment system,
you need to include a clear image of your government id on your first slide and a textual
representation of your name that can be selected and copied from your slide so that it may be pasted into a translator.
(e.g. If you use your China Resident Identity Card, you must provide clear images of the front and back
of the card. You also need to provide a textual representation of your name in Chinese characters, e.g. 蒙晶.)
\end{CJK*}

Your face must be visible throughout the presentation to show that you are the one speaking during the presentation.
This may be through Zoom's participants window.
If you cannot arrange for your face to be visible throughout the presentation,
you \textbf{must} contact the course coordinator before 28 April 2023 to discuss your constraints.

\subsection{On-Campus Identity Verification}
If you are presenting on-campus, you \textbf{must} show the marker your current and valid UQ student card.
Like in an exam situation, if you have lost your student card
you must obtain a temporary identity verification document from the UQ student centre \emph{before} the presentation.


\section{Submission}
There are four components that make up your assessable content for the presentation.
These are your draft model of the software architecture, the slides you use for your presentation,
the presentation itself, and your evaluation of other presentations.

\subsection{Draft Model}
You must show a tutor a draft architectural model of your selected system in your tutorial in week 6 (March 30).
The model must include appropriate views that give an overview of the key aspects of the system's software architecture.
You may need to give the tutor a one minute overview of the project you have selected and its key goals.
If your provided model is not an appropriate overview of the system
(e.g. too superficial, missing key parts, or too detailed)
the grade you achieve for the presentation will be reduced by one grade level.

\subsection{Slides}
The slides for your presentation are to be submitted as a PDF file to a link provided on BlackBoard.
Your slides are due at 16:00 (AEST) on the Tuesday \textbf{\textit{before}}
the practical session in which you are scheduled to make your presentation.
Late submission of your slides will result in a failing grade for the presentation.
Regardless of any penalty applied to the presentation, \emph{even} if the penalty is a failing grade,
you \textbf{\textit{must}} still make your presentation in your allocated timeslot.

\subsection{Presentation}
The presentations will take place in the practical sessions during weeks 10 to 13.
You will be allocated a week in which you are to make your presentation.
Your presentation is to use the slides you submitted on Tuesday.

If you do not deliver your presentation, your final grade will be capped at a failing grade.
If you are unable to attend your session to give your presentation due to exceptional circumstances,
you may apply to defer your presentation to another date.
You are not able to defer a deferred presentation.

\subsection{Peer Assessment}
You are expected to attend all presentations.
You are required to submit an evaluation of each presentation you observe.
Submission of \emph{meaningful} feedback for at least \textbf{75\%} of the presentations in your practical class
is required to obtain a passing grade or higher for the presentation assessment.

An online form will be provided for you to submit your assessment for each presentation.
You must submit your assessment of each presentation separately in order for the system to record all of your assessments.

If you are unable to attend a practical session due to exceptional circumstances,
and miss viewing several presentations,
you may apply for a modified limit on the number of presentations you must assess.


\section{Academic Integrity}
As this is a higher-level course, you are expected to be familiar with the importance of academic integrity in general,
and the details of UQ's rules.
If you need a reminder, review the \link{Academic Integrity Modules}
{https://web.library.uq.edu.au/library-services/it/learnuq-blackboard-help/academic-integrity-modules}.
Submissions will be checked to ensure that the work submitted is not plagiarised.
If you have quoted or paraphrased any material from another source, it must be correctly \link{cited and referenced}
{https://web.library.uq.edu.au/node/4221/2}.
Use the \link{IEEE referencing style}{https://libraryguides.vu.edu.au/ieeereferencing/gettingstarted} for citations and your bibliography.

Note that text generated by an AI tool, such as Chat GPT, is based on text from the Internet.
Consequently all text, whether written on slides or spoken during a presentation,
that was generated by an AI tool must be cited.

Uncited or unreferenced material will be treated as not being your own work.
Extensive quotation or minor rephrasing of material from cited sources should be avoided.
Significant amounts of cited material from other sources, even if paraphrased, will be considered to be of no academic merit.
In all cases, any material that you cite must support the arguments and points that you are making in your presentation.


\bibliographystyle{ieeetr}
\bibliography{ours}


\section*{Draft Model Criteria}

\begin{table}[h]
\centering
\footnotesize
\begin{tabular}{|p{2cm}|p{7.5cm}p{7.5cm}|}
\hline  & \multicolumn{2}{c|}{\textbf{Standard}}  \\ \hline
\multicolumn{1}{|c|}{\textbf{Criteria}} & \multicolumn{1}{c|}{\textbf{Acceptable}}  & \multicolumn{1}{c|}{\textbf{Not Sufficient}}  \\ \hline
\textbf{Context}               & \multicolumn{1}{p{7.5cm}|}{Provides a generally clear overview of the system.}    & System's scope and usage context are not clear.  \\ \hline
\textbf{ASRs}                  & \multicolumn{1}{p{7.5cm}|}{Identifies seemingly important goals and constraints.} & Important goals and constraints are not clear or not identified.                                                        \\ \hline
\textbf{Architecture Diagrams} & \multicolumn{1}{p{7.5cm}|}{Provide an overview of the system’s architectural structure. They also demonstrate an initial understanding of parts of the system’s internal design.} & Provides a superficial overview of the architecture structure, or architectural design is lost in system design detail. \\ \hline
\end{tabular}
\end{table}


\clearpage

\newgeometry{left=12mm,right=7mm,top=5mm,bottom=12mm}

\begin{landscape}

\fontsize{9}{11}\selectfont

\begin{xltabular}{\linewidth}{| P{1.55cm} | X | X | X | X | X | X | X |}
\hline
\multicolumn{1}{|c}{\multirow{2}{*}{\textbf{Criteria}}} &
  \multicolumn{7}{c|}{\textbf{Standard}} \\ \cline{2-8} 
\multicolumn{1}{|c}{} &
  \multicolumn{1}{c|}{\textbf{Exceptional ~ (7)}} &
  \multicolumn{1}{c|}{\textbf{Advanced ~ (6)}} &
  \multicolumn{1}{c|}{\textbf{Proficient ~ (5)}} &
  \multicolumn{1}{c|}{\textbf{Functional ~ (4)}} &
  \multicolumn{1}{c|}{\textbf{Developing ~ (3)}} &
  \multicolumn{1}{c|}{\textbf{Little Evidence ~ (2)}} &
  \multicolumn{1}{c|}{\textbf{No Evidence ~ (1)}} \\ \hline
\endhead
%
\textbf{System\newline Scope\newline20\%} &
MVP's originally proposed functional \& non-functional requirements, or those agreed \& documented early in the project, are fully delivered. &
MVP's originally proposed functional \& non-functional~require\-ments, or those agreed \& documented early in the project, are delivered with small variances. &
MVP's functional \& non-functional requirements were revised \& documented later in the project, and are almost fully delivered. &
All important functional \& non-functional requirements are delivered but some other requirements are not, whether or not original plan was revised. &
Most important functional \& non-functional requirements are delivered, whether or not original plan was revised. &
Some important functional \& non-functional requirements are delivered, whether or not original plan was revised. &
Few important functional \& non-functional requirements are delivered, whether or not original plan was revised. \\
\hline

\textbf{Architecture\newline Suitability\newline 15\%} &
Delivered architecture, supplemented by the design reflection, is very well suited to delivering all specified functional \& non-functional require\-ments, including an appropriate level of security. &
Delivered architecture, supplemented by the design reflection, is~well suited to delivering~al\-most all specified functional \& non-functional requirements, including an appropriate level of security. &
Delivered architecture, supplemented by the design reflection, is fairly well suited to delivering the key functional \& non-functional requirements, including a mostly appropriate level of security. &
Delivered architecture, supplemented by the design reflection, is capable of delivering most key functional \& non-functional requirements, including a mostly appropriate level of security. &
Delivered architecture, supplemented by the design reflection, requires workarounds in a few cases to deliver key functional \& non-functional requirements. Design has one or two obvious security issues. &
Delivered architecture, supplemented by the design reflection, requires workarounds in several cases to deliver key functional \& non-functional requirements. Design has a few obvious security issues. &
Delivered architecture, supplemented by the design reflection, makes it difficult to deliver many functional \& non-functional requirements. Design does not appear to consider security issues. \\
\hline

\textbf{Testing\newline Quality\newline 20\%} &
All functional \& non-functional requirements, \& architectural components are well tested (or are described well in a test plan) and, where feasible, are automated. &
Most key functional \& non-functional require\-ments, \& key architec\-tural components are well tested (or are described adequately in a test plan) and, where feasible, are mostly automated. &
Most key functional \& non-functional require\-ments, \& key architec\-tural components are fairly well tested (or are described fairly adequately in a test plan) and, where feasible, many are automated. &
Most key functional \& non-functional require\-ments, \& key architectural components are fairly well tested (or are described fairly adequately in a test plan) and, with some attempt at automation. &
Main test cases for most key functional \& non-functional requirements, \& key architectural components are fairly well tested (or have some informative description in a test plan). &
Main test cases for a few key functional \& non-functional requirements, \& key architectural components are moderately well tested (or have a general description in a test plan). &
Testing is poor, superficial or extremely limited. Or, extent of testing cannot be determined from submitted artefacts. \\
\hline

\textbf{Architecture\newline Description\newline 25\%} &
Clear, accurate, concise \& complete description of all aspects of the architecture. Diagrams \& narrative text complement each other. Views enhance understanding all aspects of the architecture. Choice of architecture, \& decisions about design trade-offs, are well described. &
Clear, accurate \& mostly complete description of the architecture. Diagrams \& narrative text complement each other. Views support description of the architecture. Choice of architecture, \& decisions about important design trade-offs, are well described. &
Mostly clear, accurate \& complete description of the architecture. Diagrams \& narrative text support each other. Views support some description of the architecture. Choice of architecture, \& decisions about most important design trade-offs, are adequately described. &
Fairly clear, \& mostly accurate \& complete,~des- cription of the architecture. Diagrams \&~narra- tive text are consistent. Views provide little~sup- port describing the architecture.  Choice of~archi- tecture \& decisions about some important design trade-offs, are fairly adequately described. &
Some parts of the description are unclear, in- accurate or incomplete. Most diagrams are relevant to the narrative text or a necessary diagram is missing. Justification of choice of architecture is unclear. Decisions about a few important design trade-offs are fairly adequately described. &
Some parts of the description are inaccurate or incomplete, or many parts are unclear. Some diagrams are relevant to the narrative text or a few necessary diagrams are missing. Poor justification of choice of architecture. Few design trade-offs are adequately described. &
Many parts of the description are unclear, inaccurate or incomplete. Few diagrams are relevant to the narrative text or many necessary diagrams are missing. No, or very poor, justification of choice of architecture. Trade-offs are poorly described. \\
\hline

\textbf{Architecture\newline Evaluation\newline 20\%} &
Critique \& evaluation clearly demonstrate that the delivered architecture, varied a little by the reflection comments, can deliver all functional \& non-functional requirements of the full system. &
Critique \& evaluation clearly demonstrate~that the delivered architecture, varied by the reflection comments, can deliver all functional \& non-functional requirements of the full system. &
Critique \& evaluation demonstrate that the delivered architecture, varied by the reflection comments, can deliver all important functional \& non-functional requirements of the full system. &
Critique \& evaluation demonstrate that the delivered architecture, varied by the reflection comments, can deliver all important functional \& non-functional requirements of the MVP \& part of the full system. &
Critique \& evaluation demonstrate that the delivered architecture, varied by the reflection comments, can deliver all important functional \& non-functional requirements of the MVP but little of the full system. &
Critique \& evaluation demonstrate that the delivered architecture, varied by the reflection comments, can deliver some important functional \& non-functional requirements of the MVP. &
Critique \& evaluation demonstrate that the delivered architecture, varied by the reflection comments, is unlikely to deliver most functional or non-functional requirements of the MVP. Or, they are too unclear to determine. \\
\hline

\end{xltabular}

\end{landscape}

\restoregeometry

\end{document}