\documentclass{csse4400}

\usepackage{enumitem}

\title{Capstone Project}
\author{Brae Webb}
\date{Semester 1, 2022}

\begin{document}
\maketitle

\section*{Summary}
Throughout the software architecture course,
you have learnt about a subset of quality attributes of concern to software architects.
You have also been exposed to a number of techniques to satisfy these attributes.
Now, as the capstone project, you will be required to;
\begin{itemize}
    \item propose a non-trivial software project,
    \item identify the primary quality attributes which would enable success of the project,
    \item design an architecture suitable for the aims of the project,
    \item deploy the architecture, utlizing any techniques you have learnt in or out of the course, and
    \item evaluate and report on the success of the software project.
\end{itemize}

\noindent
The successful completion of the project will result in three deliverables, namely,
\begin{enumerate}[label=\roman*]
    \item a proposal of a software project, the proposal must clearly indicate and prioritise two or three quality attributes most {\color{red} fundamental} to the project's success,
    \item the developed software, as both source code, and a deployed artifact, and
    \item a report which evaluates the success of the developed software relative to the chosen quality attributes.
\end{enumerate}

\section{Introduction}
We have looked at several core quality attributes in this course.
These attributes were picked because they are the primary concerns of many real-world software projects.
In this project, we will have an opportunity to explore some of the fun of industry.
You will take the role of an entrepreneur, software architecture, developer, and operations team.

Your first role as an entrepreneur will be to utilise some creativity and think of a software project that interests you.
Your proposed project does not have to be profitable, nor does it have to be unique.
If you are struggling to think of a project, consider what annoys you in your day-to-day life, consider if software migh help ease the annoyance.
Alternatively, look at existing everyday software like Netflix, TikTok, VSCode, or otherwise.
You are welcome to create off-brand versions of any existing software,
there are no marks for whether the software would be profitable or successful.
The lone requirement of your project is that, to function appropriately, it must demonstrate two or three of the quality attributes 
which we have already explored in class\footnote{No, simplicity is not allowed.}.

Briefly, the attributes covered so far are outlined below.
\begin{description}
    \item[Modularity] Components of the software are separated into discrete modules.
    \item[Availability] The software is available to access by end users, either at any time or on any platform, or both.
    \item[Scalability] The software is simultanoisly usable by a large amount of end users.
    \item[Extensibility] Features or extensions can be easily added to the base software.
    \item[Testibility] The software is designed so that automated tests can be easily deployed.
\end{description}

\noindent
Once you've settled on a project, write up a proposal for the project, see Section \ref{sect:proposal}.
Before you get too far writing your proposal,
please try and discuss the idea with teaching staff,
this will help ensure you don't have to re-write it from scratch.


\section{Proposal}\label{sect:proposal}
Your proposal will answer the following questions:
\begin{itemize}
    \item What is your project?
    \item Which quality attributes are most important and why?
    \item If trade offs are necessary, which attributes have higher priority?
    \item What are the basic features you plan to implement?
    \item How will you evaluate whther your project has the important quality attribute?
\end{itemize}

The proposal should not exceed two pages using 12 point sized font.
Ensure that your proposal includes a name for the project, your team name, and names and student numbers of all team members.

The suggested proposal structure is as follows.
\begin{description}
    \item[Title] A name for your project, get creative.
    \item[Authors] A name for your team.
    \item[Abstract] An elevator pitch to sell the project. This should highlight the quality attributes crucial to the project's success.
    \item[Functionality] A description of the fundamental functionality which is needed for your project. This is what you have to implement so be realistic!
    \item[Evaluation] A clear description of the way in which you will evaluate whether your project has the desired attributes.
\end{description}

\section{Software}

\section{Report}

\end{document}