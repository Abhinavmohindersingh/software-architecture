\documentclass{csse4400}

\usepackage{enumitem}

%Change space before start of \paragraph to 1.5ex.
\makeatletter
\renewcommand{\paragraph}{%
  \@startsection{paragraph}{4}%
  {\z@}{1.5ex \@plus 1ex \@minus .2ex}{-1em}%
  {\normalfont\normalsize\bfseries}%
}
\makeatother

% RUBRIC
\usepackage{multirow}
\usepackage{array}
\usepackage{xltabular}
\usepackage{pdflscape}
\usepackage{enumitem}

\newcolumntype{P}[1]{>{\centering\arraybackslash}p{#1}}
% RUBRIC

\title{Supplementary Assessment}
\author{Richard Thomas}
\date{Semester 1, 2023}

\begin{document}
\maketitle

\section*{Summary}
To assess that you have achieved the learning objectives for the course,
you need to design and evaluate the architecture for a complex system.
\begin{itemize}
    \item propose a non-trivial software project,
    \item identify the primary quality attributes which would enable success of the project,
    \item design an architecture suitable for the aims of the project,
    \item deploy the architecture, utilising any techniques you have learnt in or out of the course, and
    \item evaluate and report on the success of the software project.
\end{itemize}

\noindent
The successful completion of the project will result in three deliverables, namely,
\begin{enumerate}[label=\roman*]
    \item a proposal of a software project, the proposal must clearly indicate and prioritise two or three quality attributes most important to the project's success,
    \item the developed software, as both source code, and a deployed artifact, and
    \item a report which evaluates the success of the developed software relative to the chosen quality attributes.
\end{enumerate}

\noindent
Your software deliverable must include all supporting software (e.g. test suites or utilities) that are developed to support the delivered software.

\section{Introduction}
TradeOverflow wishes to provide a more dynamic shopping experience than traditional auction sites.
Rather than items being listed for auction over a period of time, with a fixed end date,
the sales model will be a trading platform.
TradeOverflow will scan all items that are listed for sale on the site.
It will aggregate all items of the same type.
(e.g. It will find all copies of the Dune Ultra HD Bluray that are for sale, and aggregate them into a single trading set.)

Once items have been aggregated into a trading set, the system will provide a dynamic trading platform for the item.
It is designed to promote quick sales.
Sellers list items for sale, where they set their minimum sales price.
Buyers make a purchase offer for an item, where they set their maximum purchase price.
The system continuously matches sellers and buyers.
The trading strategy is as follows.
\begin{enumerate}
	\item All purchase offers whose maximum purchase price is the same become a purchasing group.
	\item All sales listings, within a single trading set, whose minimum sales price is the same become a selling group.
	\item The purchasing group with the highest maximum purchase price becomes the active purchasing group.
	\item The selling group whose minimum sales price is closest to, but not more than, the active purchasing group's maximum purchase price becomes the active sales group.
\end{enumerate}
If the active purchasing group wants to purchase more items than are available in the active sales group, then the following trading strategy is followed.
\begin{enumerate}[resume]
	\item All items in the active sales group are sold for their minimum sales price.
	\item The items are sold in the order that the oldest purchase offer in the active purchasing group completes its purchase first.
	\item The selling group with the next closest minimum sales price that is less than the active purchasing groups' maximum purchase price becomes the active sales group.
	\item The items in this new active sales group are sold to the remaining members of the active purchasing group at the same price as items from the previous active sales group.
	\item If all items available in the new active sales group are sold, and there are still members in the active purchasing group who have not purchased an item, the process of establishing a new active sales group is repeated. This continues until either all members of the active purchasing group purchase an item, or there are no more selling groups with items whose minimum purchase price is less than or equal to the maximum purchase prices of the active purchasing group.
	\item If there are still purchasing groups who have not purchased items, the process repeats from step 3.
\end{enumerate}
If the active purchasing group wants to purchase fewer, or the same number of, items than are available in the active sales group, then the following trading strategy is followed.
\begin{enumerate}
	\setcounter{enumi}{4}
	\item The items in the active sales group are sold for the active purchasing group's maximum purchase price.
	\item The items are sold in the order that the oldest listing in the active selling group is sold first.
	\item Once all the members of the active purchasing group have purchased an item, the process repeats from step 3.
\end{enumerate}

The system needs to accommodate sellers adding new listings of an item for sale at any time.
This means that as items are listed for sale, they need to be added to the appropriate trading set and selling group.
Sellers may change the minimum sales price for which their item is listed at any time, if it has not yet been sold.
This requires their listing be moved to a different selling group.

The system also needs to accommodate members offering to purchase an item at any time.
This means that when a member submits an offer to purchase an item, it needs to be added to the appropriate purchasing group.
Purchasers may change the maximum purchase price in their offer to purchase an item at any time,
if the purchase has not yet been completed.
This requires their purchase offer be moved to a different purchasing group.

The system must guarantee that if an item sells, sellers will get at least their minimum sales price.
The system must also guarantee that, if they purchase an item, buyers will pay no more than their maximum purchase price.
If there is high demand for their item, sellers should get a higher sales price.
If there are many copies of an item for sale, buyers should get a lower purchase price.


\section{Design}
You are to design the architecture for the TradeOverflow system so that it can achieve the following key non-functional requirements. Your architecture needs to be designed so that it can be implemented using AWS services.
\begin{description}
    \item[Availability] The system must allow members to add listings or make purchase offers at any time. The system must complete trades whenever there are offers that match to available listings. The system must achieve four nines availability, that is the system must be down for less than one hour per year.
    \item[Scalability] The system must scale to meet demand. The system is expected to have peaks and troughs in trading activity. It needs to scale economically to accommodate the load. For example, popular sales periods, like Cyber Monday, may have millions of trades per hour. Whereas, other times may only have hundreds of trades per hour.
    \item[Maintainability] TradeOverflow is an ambitious company and intends to make continuous improvements to the system. The architecture and design must allow new features to be added to the system, or changes to be made to existing features, while the system is running and meets the availability and scalability non-functional requirements.
\end{description}

Your architecture is to be described through a complete set of C4 diagrams with supporting commentary.
Your C4 diagrams need to show which AWS services are being used to deliver each part of your architecture
(e.g. via the \link{Structurizr AWS theme}{https://structurizr.com/help/themes}).
The C4 diagrams should go down to the code level.
Appropriate code level diagrams to use are C4 dynamic diagrams, UML class diagrams or UML sequence diagrams.
You do not need to provide code level diagrams for the entire system,
but you do need to provide code level diagrams that demonstrate how the key features of the system will be implemented.
These key features are:
\begin{itemize}
	\item Adding new item listings to a trading set and selling group.
	\item Performing the trading strategy of when the active purchasing group wants to purchase more items than are available in the active sales group.
	\item A seller changing the minimum sales price for a listing that has not yet been sold.
\end{itemize}

You should make use of an appropriate architectural view that allows you to demonstrate how the architecture achieves the non-functional requirements described above.


\section{Evaluation}
You need to describe how you would test the TradeOverflow system to demonstrate that it delivers its key non-functional requirements.
What infrastructure and test environment would be needed to perform adequate testing?


\section{Report}
The report should include the following content.

\begin{description}
    \item[Diagrams] C4 diagrams of the system's architecture.
    \item[Architecture] Textual description of the software architecture.
    \item[Justification] Describe how the architecture delivers the key non-functional requirements. Justify the architectural choices you made in the design. Explain why the selected AWS services are suitable to deliver the key non-functional requirements.
    \item[Evaluation] Summarise how the TradeOverflow system could be tested.
\end{description}

You do not need to have sections for each topic above, though your report needs to contain the content summarised above.
For example, the architectural description and diagrams could be interleaved with each other.

Describe the architecture of the TradeOverflow system in enough detail
to give the reader a complete understanding of the architecture's design.
Describe the parts of the detailed design that demonstrate how the architecture delivers the three key features described above.


\section{Presentation}
You must complete a presentation as part of this supplementary assessment.
You cannot pass the assessment without completing the presentation.
The presentation will consist of a five minute summary of your architectural design.
You should use diagrams from your report to help the audience understand your architecture.

After your five minute summary, you will need to answer questions about your architectural design.
Questions may ask you to
\begin{itemize}[itemsep=3pt]
	\item explain parts of the detail of your architecture,
	\item justify choices made in your architectural design, or
	\item justify the technologies or services selected to deliver the architecture.
\end{itemize}


\section{Academic Integrity}
As this is a higher-level course, you are expected to be familiar with the importance of academic integrity in general, and the details of UQ's rules.
If you need a reminder, review the \link{Academic Integrity Modules}
{https://web.library.uq.edu.au/library-services/it/learnuq-blackboard-help/academic-integrity-modules}.
Submissions will be checked to ensure that the work submitted is not plagiarised.

The architecture design must be your own work and may not contain significant copied design structures.
You may find ideas for solving common architectural problems from other sources.
You must \link{cite and reference}{https://web.library.uq.edu.au/node/4221/2} these sources, and you must clearly identify what you used from those sources.
Use the \link{IEEE referencing style}{https://libraryguides.vu.edu.au/ieeereferencing/gettingstarted} for citations and references.

Uncited, unreferenced, or unacknowledged material will be treated as not being your own work.
Significant amounts of cited material from other sources will be considered to be of no academic merit.


\section{Grading Criteria}

\begin{description}[itemsep=3pt]
    \item[20\%] Extent to which project's scope was delivered.
    \item[20\%] Suitability of architecture to deliver system goals.
    \item[20\%] Quality and thoroughness of testing.
    \item[20\%] Clarity, accuracy and completeness of architecture's description.
    \item[20\%] Insightfulness of architecture's evaluation.
\end{description}


\clearpage

\newgeometry{left=12mm,right=7mm,top=5mm,bottom=12mm}

\begin{landscape}

\fontsize{9}{11}\selectfont

\begin{xltabular}{\linewidth}{| P{1.55cm} | X | X | X | X | X | X | X |}
\hline
\multicolumn{1}{|c}{\multirow{2}{*}{\textbf{Criteria}}} &
  \multicolumn{7}{c|}{\textbf{Standard}} \\ \cline{2-8} 
\multicolumn{1}{|c}{} &
  \multicolumn{1}{c|}{\textbf{Exceptional ~ (7)}} &
  \multicolumn{1}{c|}{\textbf{Advanced ~ (6)}} &
  \multicolumn{1}{c|}{\textbf{Proficient ~ (5)}} &
  \multicolumn{1}{c|}{\textbf{Functional ~ (4)}} &
  \multicolumn{1}{c|}{\textbf{Developing ~ (3)}} &
  \multicolumn{1}{c|}{\textbf{Little Evidence ~ (2)}} &
  \multicolumn{1}{c|}{\textbf{No Evidence ~ (1)}} \\ \hline
\endhead
%
\textbf{System\newline Scope\newline20\%} &
MVP's originally proposed functional \& non-functional requirements, or those agreed \& documented early in the project, are fully delivered. &
MVP's originally proposed functional \& non-functional~require\-ments, or those agreed \& documented early in the project, are delivered with small variances. &
MVP's functional \& non-functional requirements were revised \& documented later in the project, and are almost fully delivered. &
All important functional \& non-functional requirements are delivered but some other requirements are not, whether or not original plan was revised. &
Most important functional \& non-functional requirements are delivered, whether or not original plan was revised. &
Some important functional \& non-functional requirements are delivered, whether or not original plan was revised. &
Few important functional \& non-functional requirements are delivered, whether or not original plan was revised. \\
\hline

\textbf{Architecture\newline Suitability\newline 15\%} &
Delivered architecture, supplemented by the design reflection, is very well suited to delivering all specified functional \& non-functional require\-ments, including an appropriate level of security. &
Delivered architecture, supplemented by the design reflection, is~well suited to delivering~al\-most all specified functional \& non-functional requirements, including an appropriate level of security. &
Delivered architecture, supplemented by the design reflection, is fairly well suited to delivering the key functional \& non-functional requirements, including a mostly appropriate level of security. &
Delivered architecture, supplemented by the design reflection, is capable of delivering most key functional \& non-functional requirements, including a mostly appropriate level of security. &
Delivered architecture, supplemented by the design reflection, requires workarounds in a few cases to deliver key functional \& non-functional requirements. Design has one or two obvious security issues. &
Delivered architecture, supplemented by the design reflection, requires workarounds in several cases to deliver key functional \& non-functional requirements. Design has a few obvious security issues. &
Delivered architecture, supplemented by the design reflection, makes it difficult to deliver many functional \& non-functional requirements. Design does not appear to consider security issues. \\
\hline

\textbf{Testing\newline Quality\newline 20\%} &
All functional \& non-functional requirements, \& architectural components are well tested (or are described well in a test plan) and, where feasible, are automated. &
Most key functional \& non-functional require\-ments, \& key architec\-tural components are well tested (or are described adequately in a test plan) and, where feasible, are mostly automated. &
Most key functional \& non-functional require\-ments, \& key architec\-tural components are fairly well tested (or are described fairly adequately in a test plan) and, where feasible, many are automated. &
Most key functional \& non-functional require\-ments, \& key architectural components are fairly well tested (or are described fairly adequately in a test plan) and, with some attempt at automation. &
Main test cases for most key functional \& non-functional requirements, \& key architectural components are fairly well tested (or have some informative description in a test plan). &
Main test cases for a few key functional \& non-functional requirements, \& key architectural components are moderately well tested (or have a general description in a test plan). &
Testing is poor, superficial or extremely limited. Or, extent of testing cannot be determined from submitted artefacts. \\
\hline

\textbf{Architecture\newline Description\newline 25\%} &
Clear, accurate, concise \& complete description of all aspects of the architecture. Diagrams \& narrative text complement each other. Views enhance understanding all aspects of the architecture. Choice of architecture, \& decisions about design trade-offs, are well described. &
Clear, accurate \& mostly complete description of the architecture. Diagrams \& narrative text complement each other. Views support description of the architecture. Choice of architecture, \& decisions about important design trade-offs, are well described. &
Mostly clear, accurate \& complete description of the architecture. Diagrams \& narrative text support each other. Views support some description of the architecture. Choice of architecture, \& decisions about most important design trade-offs, are adequately described. &
Fairly clear, \& mostly accurate \& complete,~des- cription of the architecture. Diagrams \&~narra- tive text are consistent. Views provide little~sup- port describing the architecture.  Choice of~archi- tecture \& decisions about some important design trade-offs, are fairly adequately described. &
Some parts of the description are unclear, in- accurate or incomplete. Most diagrams are relevant to the narrative text or a necessary diagram is missing. Justification of choice of architecture is unclear. Decisions about a few important design trade-offs are fairly adequately described. &
Some parts of the description are inaccurate or incomplete, or many parts are unclear. Some diagrams are relevant to the narrative text or a few necessary diagrams are missing. Poor justification of choice of architecture. Few design trade-offs are adequately described. &
Many parts of the description are unclear, inaccurate or incomplete. Few diagrams are relevant to the narrative text or many necessary diagrams are missing. No, or very poor, justification of choice of architecture. Trade-offs are poorly described. \\
\hline

\textbf{Architecture\newline Evaluation\newline 20\%} &
Critique \& evaluation clearly demonstrate that the delivered architecture, varied a little by the reflection comments, can deliver all functional \& non-functional requirements of the full system. &
Critique \& evaluation clearly demonstrate~that the delivered architecture, varied by the reflection comments, can deliver all functional \& non-functional requirements of the full system. &
Critique \& evaluation demonstrate that the delivered architecture, varied by the reflection comments, can deliver all important functional \& non-functional requirements of the full system. &
Critique \& evaluation demonstrate that the delivered architecture, varied by the reflection comments, can deliver all important functional \& non-functional requirements of the MVP \& part of the full system. &
Critique \& evaluation demonstrate that the delivered architecture, varied by the reflection comments, can deliver all important functional \& non-functional requirements of the MVP but little of the full system. &
Critique \& evaluation demonstrate that the delivered architecture, varied by the reflection comments, can deliver some important functional \& non-functional requirements of the MVP. &
Critique \& evaluation demonstrate that the delivered architecture, varied by the reflection comments, is unlikely to deliver most functional or non-functional requirements of the MVP. Or, they are too unclear to determine. \\
\hline

\end{xltabular}

\end{landscape}

\restoregeometry

\end{document}