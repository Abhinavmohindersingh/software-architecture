\documentclass{csse4400}

\usepackage{enumitem}

%Change space before start of \paragraph to 1.5ex.
\makeatletter
\renewcommand{\paragraph}{%
  \@startsection{paragraph}{4}%
  {\z@}{1.5ex \@plus 1ex \@minus .2ex}{-1em}%
  {\normalfont\normalsize\bfseries}%
}
\makeatother

% RUBRIC
\usepackage{multirow}
\usepackage{array}
\usepackage{xltabular}
\usepackage{pdflscape}
\usepackage{enumitem}

\newcolumntype{P}[1]{>{\centering\arraybackslash}p{#1}}
% RUBRIC

\title{Supplementary Assessment}
\author{Richard Thomas}
\date{Semester 1, 2023}

\begin{document}
\maketitle

\section*{Summary}
To assess that you have achieved the learning objectives for the course,
you need to design and evaluate the architecture for a complex system.
\begin{itemize}
    \item propose a non-trivial software project,
    \item identify the primary quality attributes which would enable success of the project,
    \item design an architecture suitable for the aims of the project,
    \item deploy the architecture, utilising any techniques you have learnt in or out of the course, and
    \item evaluate and report on the success of the software project.
\end{itemize}

\noindent
The successful completion of the project will result in three deliverables, namely,
\begin{enumerate}[label=\roman*]
    \item a proposal of a software project, the proposal must clearly indicate and prioritise two or three quality attributes most important to the project's success,
    \item the developed software, as both source code, and a deployed artifact, and
    \item a report which evaluates the success of the developed software relative to the chosen quality attributes.
\end{enumerate}

\noindent
Your software deliverable must include all supporting software (e.g. test suites or utilities) that are developed to support the delivered software.

\section{Introduction}
eBay\texttrademark wishes to provide a more dynamic shopping experience.
Rather than items being listed for auction over a period of time, with a fixed end date,
the new sales model will be a trading platform.
The system will scan all items that are listed for auction on the site.
It will aggregate all items of the same type.
(e.g. It will find all copies of the Dune Ultra HD Bluray that are for sale, and aggregate them into a single trading set.)

Once items have been aggregated into a trading set, the system will provide a dynamic trading platform for the item.
It is designed to promote quick sales.
Sellers set their minimum sales price.
Buyers set their maximum purchase price.
The system continuously matches sellers and buyers.
The strategy is as follows.
\begin{enumerate}
	\item All buyers whose maximum purchase price is the same become a purchasing group.
	\item All sellers whose minimum sales price is the same become a selling group.
	\item The purchasing group with the highest maximum purchase price becomes the active purchasing group.
	\item The selling group whose minimum sales price is closest to, but not more than, the active purchasing group's maximum purchase price becomes the active sales group.
\end{enumerate}
If the active purchasing group wants to purchase more items than are available in the active sales group, then the following sales strategy is followed.
\begin{enumerate}
	\item All items in the active sales group are sold for their minimum sales price.
	\item The items are sold in the order that the oldest purchase offer at the purchasing group's maximum purchase price completes its purchase first.
	\item The selling group with the next closest minimum sales price that is less than the active purchasing groups' maximum purchase price becomes the active sales group.
	\item The items in this new active sales group are sold to the remaining members of the active purchasing group at the same price as items from the previous active sales group.
	\item If all items available in the new active sales group are sold, and there are still members in the active purchasing group who have not purchased an item, the process of establishing a new active sales group is repeated. This continues until either all members of the active purchasing group purchase an item, or there are no more selling groups with items whose minimum purchase price is less than or equal to the maximum purchase prices of the active purchasing group.
	\item If there are still purchasing groups who have not purchased items, the process repeats from step 3.
\end{enumerate}
If the active purchasing group wants to purchase fewer, or the same number of, items than are available in the active sales group, then the following sales strategy is followed.
\begin{enumerate}
	\item The items in the active sales group are sold for the active purchasing group's maximum purchase price.
	\item The items are sold in the order that the oldest item listing at the selling group's minimum sales price is sold first.
\end{enumerate}

The system guarantees that if the item sells, sellers will get at least their minimum sales price.
The system also guarantees that if they purchase an item, that buyers will pay no more than their maximum purchase price.
If there is high demand for their item, sellers should get a higher sales price.
If there are many copies of an item for sale, buyers should get a lower purchase price.


You are to design the architecture for 


\section{Software}
You need to implement a software system that delivers a \link{Minimum Viable Product (MVP)}{https://www.agilealliance.org/glossary/mvp/}.
The MVP needs to implement a usable core of the system's functionality,
which demonstrates that the architecture could deliver the full system functionality.
The MVP also needs to allow the software architecture to be tested to determine if it can deliver the project's important quality attributes.

You may renegotiate the scope of the system during the project,
if you determine that certain aspects of the original scope are not feasible within the project time constraints.
The earlier you do this, the less it will impact on your final result.
You will not explicitly lose marks for renegotiating scope, unless the revised scope limits your ability to adequately test important quality attributes.
But, late changes to scope are likely to have a flow-on effect that could reduce the quality of your final deliverables.
This means that you should attempt to implement some of the riskier parts of the project early.


\section{Evaluation}
You need to test the software system that you implement to demonstrate
how well its architecture supports delivering system functionality and its quality attributes.
This evaluation should be based on the proposal's evaluation plan, but should \textbf{\emph{not}} be limited to only what is in that plan.
You will be assessed on how well you test your system in terms of both functionality and quality attributes.
Discovering issues with the system or its architecture during testing will not adversely affect your marks for the evaluation component of the assessment.

A section of your project report needs to summarise the test results and provide access to the full suite of tests.
You should automate as much of the testing as possible.
Any manual tests need to be documented so that they can be duplicated.
The results of all manual tests need to be recorded in a test report.
This may be a section of the project report, or it may be a separate document with a link to it from the project report.
You need to include test code and test infrastructure in your project's repository.


\section{Report}
The report should include the following content.

\begin{description}
    \item[Title] Name of your software project.
    \item[Abstract] Summarise the key points of your document.
    \item[Changes] Describe and justify any changes made to the project from what was outlined in the proposal.
    \item[Architecture] Describe the MVP's software architecture in detail.
    \item[Trade-Offs] Describe and justify the trade-offs made in designing the architecture.
    \item[Critique] Describe how well the architecture supports delivering the complete system.
    \item[Evaluation] Summarise testing results and justify how well the software achieves its quality attributes.
    \item[Reflection] Lessons learnt and what you would do differently.
\end{description}

You do not need to have sections for each topic above, though your report needs to contain the content summarised above.
For example, the description of the architecture could include discussion of trade-offs.
Similarly, the critique and evaluation could be combined so that both are discussed in relation to an ASR.

When writing your report, you may assume that the reader is familiar with the project proposal.
You will need to describe any changes your team has made to the original proposal.
A rationale should be provided for each change.
Small changes only need a brief summary of the reason for the change.
Significant changes to functionality of the MVP, or changes to important quality attributes,
need a more detailed justification for the change.
You should provide a reference and link to the original proposal.

Describe the full architecture of your MVP in enough detail
to give the reader a complete understanding of the architecture's design.
Use appropriate views, diagrams and commentary to describe the software architecture.
You should describe parts of the detailed design that demonstrate how the architecture supports delivering key quality attributes.
(e.g. If interoperability was a key quality attribute, you would need to describe the parts of the detailed design that support this.
For example, how you use the adapter design pattern to communicate with external services.)

Describe any trade-offs made during the design of the architecture.
Explain what were the competing issues%
\footnote{``\href{http://www.cs.unc.edu/~stotts/COMP723-s15/patterns/forces.html}{Forces}'' in design patterns terminology.}
and explain why you made the decisions that resulted in your submitted design.

When describing the architecture and trade-offs,
you should summarise and/or reference ADRs that relate to important decisions that affected your architecture.

Your critique should discuss how well the architecture is suited to delivering the full system functionality and quality attributes.
Use test results to support your claims, where this can be shown through testing.
For quality attributes that cannot be easily tested (e.g. extensibility, interoperability, ...),
you will need to provide an argument, based on your architectural design, about how the design supports or enables the attribute.
Some quality attributes (e.g. scalability) may require both test results
and argumentation to demonstrate how well the attributed is delivered.

Summarise test plans and test results in the report.
Provide links to any test plans, scripts or code in your repository.
Where feasible, tests should be automated.
Describe how to run the tests.
Ideally, you should use \link{GitHub Actions}{https://docs.github.com/en/actions}
to run tests and potentially deploy artefacts.

Your report should end with a reflection that summarises what you have learnt from designing and implementing this project.
It should include descriptions of what you would do differently, after the experience of implementing the project.
Describe potential benefits or improvements that may be delivered by applying the lessons you have learnt during the project.


\section{Repository}
Your team will be provisioned with a repository on GitHub.
All development work and documentation are to be committed to the repository.
All project artefacts are to be submitted via this repository.

\begin{itemize}
    \item Model artefacts (e.g. Structurizr DSL or PUML files) should be stored in the \textbf{\texttt{/model}} directory.
    \item ADRs are to be stored in the \textbf{\texttt{/model/adrs}} directory.
    \item The report must be stored in the \textbf{\texttt{/report}} directory.
    \item The link to your demonstration video must be in a file called \textbf{\texttt{demo.md}}.
\end{itemize}

Do \textbf{\emph{not}} commit large binary files to the repository.
(i.e. Do not commit Word documents or frequently changing PDF files to the repository.
Do not store your demonstration video in your repository.)
It is recommended that you use LaTeX, or possibly markdown, to write your report.
If you use LaTeX, you should use GitHub actions to produce a PDF of the report.

Your final submission will be what is in your repository at the due date of 16:00 (AEST) on 5 June 2023.


\section{Academic Integrity}
As this is a higher-level course, you are expected to be familiar with the importance of academic integrity in general, and the details of UQ's rules.
If you need a reminder, review the \link{Academic Integrity Modules}
{https://web.library.uq.edu.au/library-services/it/learnuq-blackboard-help/academic-integrity-modules}.
Submissions will be checked to ensure that the work submitted is not plagiarised.

All code that you submit must be your own work or must be appropriately cited.
If you find ideas, code fragments, or libraries from external sources (e.g. Stack Overflow), you must \link{cite and reference}{https://web.library.uq.edu.au/node/4221/2} these sources.
Use the \link{IEEE referencing style}{https://libraryguides.vu.edu.au/ieeereferencing/gettingstarted} for citations and references.
Citations should be included in a comment at the location where the idea is used in your code.
All references for citations must be included in a file called \texttt{refs.md}.
This file should be in the root directory of your repository.

You are encouraged to use a generative AI tool (e.g. copilot) to help you write the source code for this project.
The expectation is that the software architecture and detailed design are your team's own work.
Create a file in the root directory of your repository called \texttt{ai.md}.
In \texttt{ai.md} indicate which files contain code produced with the assistance of an AI tool.
Estimate how much of the code was produced by the tool and how much was your own work
(e.g. \texttt{logic.py ~~~40\% generated}).

You may use libraries to help implement your project.
The library's license must allow you to use it in the context of your project.
All libraries used in your project must be listed in a file called \texttt{libs.md}.
This file must be in the root directory of your repository.

Uncited, unreferenced or unacknowledged material will be treated as not being your own work.
Significant amounts of cited material from other sources will be considered to be of no academic merit.
Having an AI tool produce significant amounts of source code is acceptable,
if the design is your own and you have verified that the code is correct.


\section{Demonstration}
Your team needs to demonstrate your project's functionality and how well it achieves its goals.
This should include demonstrating how quality attributes are achieved,
or briefly summarising how the architecture facilitates delivering a quality attribute.

Your project demonstration will be a video.
Provide a link to the video in a file called \textbf{\texttt{demo.md}}, stored in the root directory of your repository.
Do \textbf{\textit{not}} store the video in your GitHub repository.
The link may be to the video on a platform like YouTube, or a file sharing site from where the video may be downloaded.
If you upload the video to a platform like YouTube, you may make it private.
If it is a private video, you must share it with \texttt{richard.thomas@uq.edu.au}, \texttt{b.webb@uq.edu.au},
\texttt{evan.hughes@uq.edu.au}, \texttt{m.holloway@uq.\textbf{net}.au}, and all of your team members.
The video must remain available until at least 31 July 2023.
Viewers must be able to easily see what is being demonstrated and read any text or images.
Audio must be clear and comprehensible.

The video timeline should be as follows.

\begin{description}
    \item[3 min] Introduction to the project.
    \item[3 min] Demonstration of the functional requirements.
    \item[3 min] Demonstration of the non-functional requirements.
    \item[3 min] Overview of the software architecture.
    \item[3 min] Summary of your reflection on lessons learnt from implementing the software.
\end{description}

\noindent
The total duration of your video should be \textbf{\emph{less}} than 15 minutes.

\paragraph{Introduction} Briefly introduce the project context and summarise the delivered functional and non-functional requirements.
Mention any differences between what was originally proposed, what was renegotiated, and what was delivered.
Briefly explain why changes in scope were made.
The tutor who may have approved a change in scope may not be the tutor marking your demonstration.
If you did not deliver everything in the revised scope of the project, the marker needs to know why that occurred.

\paragraph{Functional Requirements} Demonstrate the key features of the software.
You do not have time to demonstrate every feature of the software.
Plan your time wisely to highlight the completeness and quality of your delivered system.

\paragraph{Non-Functional Requirements} Show how well the software delivers its important quality attributes.
This may take some thought and planning to demonstrate within a short time frame.
Delivery of some non-functional requirements can be shown by test results.
Delivery of other non-functional requirements may be shown through a combination of tests, metrics, and commentary.

For example, you cannot show ten minutes of k6 testing to demonstrate scalability.
You could provide screenshots of different stages of the testing, or an edited video of parts of the testing.
You would provide commentary summarising how the testing was done and explaining how well the system scaled under different loads.

For security, you could show results of simple fuzz testing of APIs.
You could then show examples of parts of your design, explaining how it demonstrates following key security design principles.

For extensibility or interoperability, you could calculate one or more complexity metrics for parts of the design.
You could then use the data from these metrics to support an argument as to why the design was extensible or had a simple interface.
For example, if many interfaces could be shown to have high cohesion and there was low coupling between different modules
or services in the design, you could argue how this shows that the design is likely to be extensible.
You could measure documentation for interfaces or APIs,
and use that to argue that mechanisms used to extend the design, or that the APIs, were comprehensible.

These are examples to help you to start thinking about demonstrating how your design delivers non-functional requirements.
They are not a definitive list of the only or best approaches.
For the demonstration, focus on the most important non-functional requirements for your project.
You should discuss your ideas with a tutor if you are unsure of the effectiveness of an approach.

\paragraph{Software Architecture} Provide an overview the system's architecture.
Briefly explain how well it supports delivery of the MVP's, and the full system's, functional and non-functional requirements.

\paragraph{Reflection} Summarise the lessons you learnt from implementing the software.
What would you do differently and why?
Explain how you would apply those lessons to design a different architecture or take a different approach to implementing the project.
Or, explain how the lessons learnt demonstrate that you made good choices at each stage of development.

\paragraph{Presentation} There are no constraints on who in your team presents in the video.
One person could present all parts of the video, or you could have different people presenting each part.
Assume that the viewer has read the project proposal but may not yet have read the project report.

%You may use this \link{booking page}{https://calendly.com/richard-thomas-uq/csse6400-demo}
%to schedule a time to demonstrate your project between June 10 and 17.
%The demonstration is to take a \textbf{maximum} of 20 minutes.
%Your system must be deployed and set up so you can start your demonstration \textbf{immediately}.
%You should not take demonstration time to do any deployment or initialisation.
%
%When you book the demonstration time, you will be sent a confirmation email with a Zoom link for the demonstration.
%Only \textbf{one person} from your team should make a demonstration booking.
%Ensure you discuss with your other team members to find suitable times that are available for booking,
%and for which at least all of your key team members are available.


\section{Grading Criteria}

\begin{description}[itemsep=3pt]
    \item[20\%] Extent to which project's scope was delivered.
    \item[20\%] Suitability of architecture to deliver system goals.
    \item[20\%] Quality and thoroughness of testing.
    \item[20\%] Clarity, accuracy and completeness of architecture's description.
    \item[20\%] Insightfulness of architecture's evaluation.
\end{description}


\clearpage

\newgeometry{left=12mm,right=7mm,top=5mm,bottom=12mm}

\begin{landscape}

\fontsize{9}{11}\selectfont

\begin{xltabular}{\linewidth}{| P{1.55cm} | X | X | X | X | X | X | X |}
\hline
\multicolumn{1}{|c}{\multirow{2}{*}{\textbf{Criteria}}} &
  \multicolumn{7}{c|}{\textbf{Standard}} \\ \cline{2-8} 
\multicolumn{1}{|c}{} &
  \multicolumn{1}{c|}{\textbf{Exceptional ~ (7)}} &
  \multicolumn{1}{c|}{\textbf{Advanced ~ (6)}} &
  \multicolumn{1}{c|}{\textbf{Proficient ~ (5)}} &
  \multicolumn{1}{c|}{\textbf{Functional ~ (4)}} &
  \multicolumn{1}{c|}{\textbf{Developing ~ (3)}} &
  \multicolumn{1}{c|}{\textbf{Little Evidence ~ (2)}} &
  \multicolumn{1}{c|}{\textbf{No Evidence ~ (1)}} \\ \hline
\endhead
%
\textbf{System\newline Scope\newline20\%} &
MVP's originally proposed functional \& non-functional requirements, or those agreed \& documented early in the project, are fully delivered. &
MVP's originally proposed functional \& non-functional~require\-ments, or those agreed \& documented early in the project, are delivered with small variances. &
MVP's functional \& non-functional requirements were revised \& documented later in the project, and are almost fully delivered. &
All important functional \& non-functional requirements are delivered but some other requirements are not, whether or not original plan was revised. &
Most important functional \& non-functional requirements are delivered, whether or not original plan was revised. &
Some important functional \& non-functional requirements are delivered, whether or not original plan was revised. &
Few important functional \& non-functional requirements are delivered, whether or not original plan was revised. \\
\hline

\textbf{Architecture\newline Suitability\newline 15\%} &
Delivered architecture, supplemented by the design reflection, is very well suited to delivering all specified functional \& non-functional require\-ments, including an appropriate level of security. &
Delivered architecture, supplemented by the design reflection, is~well suited to delivering~al\-most all specified functional \& non-functional requirements, including an appropriate level of security. &
Delivered architecture, supplemented by the design reflection, is fairly well suited to delivering the key functional \& non-functional requirements, including a mostly appropriate level of security. &
Delivered architecture, supplemented by the design reflection, is capable of delivering most key functional \& non-functional requirements, including a mostly appropriate level of security. &
Delivered architecture, supplemented by the design reflection, requires workarounds in a few cases to deliver key functional \& non-functional requirements. Design has one or two obvious security issues. &
Delivered architecture, supplemented by the design reflection, requires workarounds in several cases to deliver key functional \& non-functional requirements. Design has a few obvious security issues. &
Delivered architecture, supplemented by the design reflection, makes it difficult to deliver many functional \& non-functional requirements. Design does not appear to consider security issues. \\
\hline

\textbf{Testing\newline Quality\newline 20\%} &
All functional \& non-functional requirements, \& architectural components are well tested (or are described well in a test plan) and, where feasible, are automated. &
Most key functional \& non-functional require\-ments, \& key architec\-tural components are well tested (or are described adequately in a test plan) and, where feasible, are mostly automated. &
Most key functional \& non-functional require\-ments, \& key architec\-tural components are fairly well tested (or are described fairly adequately in a test plan) and, where feasible, many are automated. &
Most key functional \& non-functional require\-ments, \& key architectural components are fairly well tested (or are described fairly adequately in a test plan) and, with some attempt at automation. &
Main test cases for most key functional \& non-functional requirements, \& key architectural components are fairly well tested (or have some informative description in a test plan). &
Main test cases for a few key functional \& non-functional requirements, \& key architectural components are moderately well tested (or have a general description in a test plan). &
Testing is poor, superficial or extremely limited. Or, extent of testing cannot be determined from submitted artefacts. \\
\hline

\textbf{Architecture\newline Description\newline 25\%} &
Clear, accurate, concise \& complete description of all aspects of the architecture. Diagrams \& narrative text complement each other. Views enhance understanding all aspects of the architecture. Choice of architecture, \& decisions about design trade-offs, are well described. &
Clear, accurate \& mostly complete description of the architecture. Diagrams \& narrative text complement each other. Views support description of the architecture. Choice of architecture, \& decisions about important design trade-offs, are well described. &
Mostly clear, accurate \& complete description of the architecture. Diagrams \& narrative text support each other. Views support some description of the architecture. Choice of architecture, \& decisions about most important design trade-offs, are adequately described. &
Fairly clear, \& mostly accurate \& complete,~des- cription of the architecture. Diagrams \&~narra- tive text are consistent. Views provide little~sup- port describing the architecture.  Choice of~archi- tecture \& decisions about some important design trade-offs, are fairly adequately described. &
Some parts of the description are unclear, in- accurate or incomplete. Most diagrams are relevant to the narrative text or a necessary diagram is missing. Justification of choice of architecture is unclear. Decisions about a few important design trade-offs are fairly adequately described. &
Some parts of the description are inaccurate or incomplete, or many parts are unclear. Some diagrams are relevant to the narrative text or a few necessary diagrams are missing. Poor justification of choice of architecture. Few design trade-offs are adequately described. &
Many parts of the description are unclear, inaccurate or incomplete. Few diagrams are relevant to the narrative text or many necessary diagrams are missing. No, or very poor, justification of choice of architecture. Trade-offs are poorly described. \\
\hline

\textbf{Architecture\newline Evaluation\newline 20\%} &
Critique \& evaluation clearly demonstrate that the delivered architecture, varied a little by the reflection comments, can deliver all functional \& non-functional requirements of the full system. &
Critique \& evaluation clearly demonstrate~that the delivered architecture, varied by the reflection comments, can deliver all functional \& non-functional requirements of the full system. &
Critique \& evaluation demonstrate that the delivered architecture, varied by the reflection comments, can deliver all important functional \& non-functional requirements of the full system. &
Critique \& evaluation demonstrate that the delivered architecture, varied by the reflection comments, can deliver all important functional \& non-functional requirements of the MVP \& part of the full system. &
Critique \& evaluation demonstrate that the delivered architecture, varied by the reflection comments, can deliver all important functional \& non-functional requirements of the MVP but little of the full system. &
Critique \& evaluation demonstrate that the delivered architecture, varied by the reflection comments, can deliver some important functional \& non-functional requirements of the MVP. &
Critique \& evaluation demonstrate that the delivered architecture, varied by the reflection comments, is unlikely to deliver most functional or non-functional requirements of the MVP. Or, they are too unclear to determine. \\
\hline

\end{xltabular}

\end{landscape}

\restoregeometry


\bibliographystyle{ieeetr}
\bibliography{ours}

\end{document}