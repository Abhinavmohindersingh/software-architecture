\documentclass{csse4400}

\usepackage{enumitem}
\usepackage{fancyhdr}

%Change space before start of \paragraph to 1.5ex.
\makeatletter
\renewcommand{\paragraph}{%
  \@startsection{paragraph}{4}%
  {\z@}{1.5ex \@plus 1ex \@minus .2ex}{-1em}%
  {\normalfont\normalsize\bfseries}%
}
\makeatother


\title{Supplementary Assessment}
\author{Richard Thomas}
\date{Semester 1, 2023}


\begin{document}

% Custom footer with UQ copyright notice to facilitate takedown requests at academic file sharing sites (e.g. Course Hero or Chegg).
% Requires document to \usepackage{fancyhdr}.

\pagestyle{fancy}

% Remove all default header content.
\fancyhead{} % Clear default header fields.
\renewcommand{\headrulewidth}{0pt} % Remove horizontal rule from header.

% Set footer details.
\setlength{\footskip}{15mm}
\fancyfoot{}  % Clear default footer fields.
\fancyfoot[L]{\small \copyright \ The University of Queensland \the\year\ }
\fancyfoot[R]{\small Page \thepage}

\maketitle

\section*{Summary}
To assess that you have achieved the learning objectives for the course,
you need to design and evaluate the architecture for a complex system.


\section{Introduction}
TradeOverflow wishes to provide a more dynamic shopping experience than traditional auction sites.
Rather than items being listed for auction over a period of time, with a fixed end date,
the sales model will be a trading platform.
TradeOverflow will scan all items that are listed for sale on the site.
It will aggregate all items of the same type into a trading set.
(e.g. It will find all copies of the Dune Ultra HD Bluray that are for sale, and aggregate them into a single trading set.)

Once items have been aggregated into a trading set, the system will provide a dynamic trading platform for the item.
It is designed to promote quick sales.
Sellers list items for sale, where they set their minimum sales price.
Buyers make a purchase offer for an item, where they set their maximum purchase price.
The system continuously matches sellers and buyers.
The trading strategy is as follows.
\begin{enumerate}
	\item All purchase offers whose maximum purchase price is the same become a purchasing group.
	\item All sales listings, within a single trading set, whose minimum sales price is the same become a selling group.
	\item The purchasing group with the highest maximum purchase price becomes the active purchasing group.
	\item The selling group whose minimum sales price is closest to, but not more than, the active purchasing group's maximum purchase price becomes the active sales group.
\end{enumerate}
If the active purchasing group wants to purchase more items than are available in the active sales group, then the following trading strategy is followed.
\begin{enumerate}[resume]
	\item All items in the active sales group are sold for their minimum sales price.
	\item The items are sold in the order that the oldest purchase offer in the active purchasing group completes its purchase first.
	\item The selling group with the next closest minimum sales price that is less than the active purchasing groups' maximum purchase price becomes the active sales group.
	\item The items in this new active sales group are sold to the remaining members of the active purchasing group at the same price as items from the previous active sales group.
	\item If all items available in the new active sales group are sold, and there are still members in the active purchasing group who have not purchased an item, the process of establishing a new active sales group is repeated. This continues until either all members of the active purchasing group purchase an item, or there are no more selling groups with items whose minimum purchase price is less than or equal to the maximum purchase prices of the active purchasing group.
	\item If there are still purchasing groups who have not purchased items, the process repeats from step 3.
\end{enumerate}
If the active purchasing group wants to purchase fewer, or the same number of, items than are available in the active sales group, then the following trading strategy is followed.
\begin{enumerate}
	\setcounter{enumi}{4}
	\item The items in the active sales group are sold for the active purchasing group's maximum purchase price.
	\item The items are sold in the order that the oldest listing in the active selling group is sold first.
	\item Once all the members of the active purchasing group have purchased an item, the process repeats from step 3.
\end{enumerate}

The system needs to accommodate sellers adding new listings of an item for sale at any time.
This means that as items are listed for sale, they need to be added to the appropriate trading set and selling group.
Sellers may change the minimum sales price for which their item is listed at any time, if it has not yet been sold.
This requires their listing be moved to a different selling group.

The system also needs to accommodate members offering to purchase an item at any time.
This means that when a member submits an offer to purchase an item, it needs to be added to the appropriate purchasing group.
Purchasers may change the maximum purchase price in their offer to purchase an item at any time,
if the purchase has not yet been completed.
This requires their purchase offer be moved to a different purchasing group.

The system must guarantee that if an item sells, sellers will get at least their minimum sales price.
The system must also guarantee that, if they purchase an item, buyers will pay no more than their maximum purchase price.
If there is high demand for their item, sellers should get a higher sales price.
If there are many copies of an item for sale, buyers should get a lower purchase price.


\section{Design}
You are to design the architecture for the TradeOverflow system so that it can achieve the following key non-functional requirements. Your architecture needs to be designed so that it can be implemented using AWS services.
\begin{description}
    \item[Availability] The system must allow members to add listings or make purchase offers at any time. The system must complete trades whenever there are offers that match to available listings. The system must achieve four nines availability, that is the system must be down for less than one hour per year.
    \item[Scalability] The system must scale to meet demand. The system is expected to have peaks and troughs in trading activity. It needs to scale economically to accommodate the load. For example, popular sales periods, like Cyber Monday, may have millions of trades per hour. Whereas, other times may only have hundreds of trades per hour.
    \item[Maintainability] TradeOverflow is an ambitious company and intends to make continuous improvements to the system. The architecture and design must allow new features to be added to the system, or changes to be made to existing features, while the system is running and meets the availability and scalability non-functional requirements.
\end{description}

Your architecture is to be described through a complete set of C4 diagrams with supporting commentary.
Your C4 diagrams need to show which AWS services are being used to deliver each part of your architecture
(e.g. via the \link{Structurizr AWS theme}{https://structurizr.com/help/themes}).
The C4 diagrams should go down to the code level.
Appropriate code level diagrams to use are C4 dynamic diagrams, UML class diagrams or UML sequence diagrams.
You do not need to provide code level diagrams for the entire system,
but you do need to provide code level diagrams that demonstrate how the key features of the system will be implemented.
These key features are:
\begin{itemize}
	\item Adding new item listings to a trading set and selling group.
	\item Performing the trading strategy of when the active purchasing group wants to purchase more items than are available in the active sales group.
	\item A seller changing the minimum sales price for a listing that has not yet been sold.
\end{itemize}
You should make use of an appropriate architectural view that allows you to demonstrate how the architecture achieves the non-functional requirements described above.


\section{Evaluation}
You need to describe how you would test the TradeOverflow system to demonstrate that it delivers its key non-functional requirements.
This needs to describe the infrastructure and test environment that would be needed to perform adequate testing.
You need to justify why the proposed testing strategy would be sufficient to provide confidence that the system meets its functional and non-functional requirements.


\section{Report}
You need to submit a report that includes the following content.
\begin{description}
    \item[Diagrams] C4 diagrams of the system's architecture.
    \item[Architecture] Textual description of the software architecture.
    \item[Justification] Describe how the architecture delivers the key non-functional requirements. Justify the architectural choices you made in the design. Explain why the selected AWS services are suitable to deliver the key non-functional requirements.
    \item[Security] Describe the potential digital attack surface of the system and how it should be designed and implemented to reduce the risk of unauthorised access to the system.
    \item[Evaluation] Summarise how the TradeOverflow system could be tested.
\end{description}
You do not need to have sections for each topic above, though your report needs to contain the content summarised above.
For example, the architectural description and diagrams could be interleaved with each other.

Describe the architecture of the TradeOverflow system in enough detail
to give the reader a complete understanding of the architecture's design.
Describe the parts of the detailed design that demonstrate how the architecture delivers the three key features described above.


\section{Presentation}
You must complete a presentation as part of this supplementary assessment.
You cannot pass the assessment without completing the presentation.
The presentation will consist of a six minute summary of your architectural design and justification of your design choices.
You should use diagrams from your report to help the audience understand your architecture.

After your summary presentation, you will need to answer questions about your architectural design.
Questions may ask you to
\begin{itemize}
	\item explain parts of the detail of your architecture,
	\item justify choices made in your architectural design, or
	\item justify the technologies or services selected to deliver the architecture.
\end{itemize}


\section{Submission}
\begin{itemize}
	\item Submit your report by 4pm on 18 August 2023 through the Supplementary Assessment TurnItIn link on the CSSE6400 BlackBoard site.
	\item Arrange an appointment to make your presentation by emailing \href{mailto:richard.thomas@uq.edu.au}{richard.thomas@uq.edu.au} by 4pm on 18 August 2023.
	\item Make your presentation in-person by 4pm on 22 August 2023, unless external factors make that impossible, in which case the presentation will be made over Zoom.
\end{itemize}


\section{Academic Integrity}
As this is a higher-level course, you are expected to be familiar with the importance of academic integrity in general, and the details of UQ's rules.
If you need a reminder, review the \link{Academic Integrity Modules}
{https://web.library.uq.edu.au/library-services/it/learnuq-blackboard-help/academic-integrity-modules}.
Submissions will be checked to ensure that the work submitted is not plagiarised.

The architecture design must be your own work and may not contain significant copied design structures.
You may find ideas for solving common architectural problems from other sources.
You must \link{cite and reference}{https://web.library.uq.edu.au/node/4221/2} these sources, and you must clearly identify what you used from those sources.
Use the \link{IEEE referencing style}{https://libraryguides.vu.edu.au/ieeereferencing/gettingstarted} for citations and references.

Uncited, unreferenced, or unacknowledged material will be treated as not being your own work.
Significant amounts of cited material from other sources will be considered to be of no academic merit.


\section{Grading Criteria}
This is a pass/fail assessment.
Your result will either be a passing or failing grade for the course.
You must pass \textbf{all} of the following criteria.
\begin{description}
    \item[Architecture] The described architecture is capable of delivering the system's functional and non-functional requirements.
    \item[Technology] The selected AWS services are capable of achieving the non-functional requirements of the system and the architecture structures these in a fashion that will work.
    \item[Justification] You demonstrate appropriate consideration of trade-offs when selecting services and making architectural design choices.
    \item[Security] You demonstrate adequate knowledge of potential security risks in the architecture and how these may be managed.
    \item[Evaluation] You demonstrate adequate knowledge of how to test a complex distributed system.
    \item[Presentation] You demonstrate in-depth knowledge of the architecture design and its rationale.
\end{description}


\end{document}