\documentclass{csse4400}

\hypersetup{
    colorlinks=true,
    linkcolor=violet,
    filecolor=purple,      
    urlcolor=blue,
}

\title{Documenting an Architecture}
\author{Brae Webb}
\date{Semester 1, 2022}

\begin{document}
\maketitle

\section*{Summary}
In this assignment, you will be asked to demonstrate your ability to
\textsl{understand} and subsequently \textsl{communicate} an architecture of an existing software project.
\begin{enumerate}
\item First, you need to choose a suitable open source software project.
        The project must have non-trivial functionality and architecture.
\item You will then need to write a 2-4 page markdown document (excluding appendices) which 
        proficiently documents the architecture of your software project.
\end{enumerate}

\section{Introduction}
The digital world relies heavily on open source software, as seen by the recent log4j vulnerability.\footnote{https://www.cisa.gov/uscert/apache-log4j-vulnerability-guidance}
Fortunately, open source developers often maintain high quality documentation for the users of their projects.
Unfortunately however, many open source projects don't maintain the same high quality documentation for the architecture of their software projects.
This can cause difficulty for developers who want to contribute to the project, but first need to understand it.

In this project, you have the chance to right this wrong.
Your task is to find an open source software project with a sufficiently complex architecture and document it.
You may optionally choose to share this documentation with the project developers.
You are encouraged to do this, as the perspective of a newcomer to a project is often invaluable to the seasoned developers.

Before looking for projects, read some of the architecture documentation written by students at TU Delft:
\url{https://delftswa.gitbooks.io/desosa2016/content} and \url{https://delftswa.github.io}.

It would also be advantageous to read through one of the architecture descriptions in either volume of
The Architecture of Open Source Applications: \url{http://aosabook.org}.

\section{Finding a Project}
Criteria for the software project:
\begin{itemize}
    \item The project cannot be covered in the tutorials or by the TU Delft students above.
    \item The project must have least one release within the last year.
\end{itemize}

\noindent Places to look for projects:
\begin{itemize}
    \item GitHub explore page: \url{https://github.com/explore};
    \item Apache project list: \url{https://apache.org/index.html#projects-list};
    \item in class discussion with other students;
    \item or, ask your tutor.
\end{itemize}

\section{Documentation Structure}

\begin{description}
    \item[Title] Name of the software project.
    \item[Abstract] Summarise the key points of your document.
    \item[Introduction] Describe the software project, explaining the its key functionality and target users.
    \item[Context] Provide an overview of the software system's context and its external dependencies.
    \item[Architecture] Describe the software's architecture.
    \item[Critique] Analyse the software's architecture, describing its advantages and disadvantages.
    \item[Conclusion] Highlight the key points or lessons learnt about the software's architecture.
\end{description}

\section{Documentation Content}
How you present the information in your documentation, is up to you. You will need to select an appropriate notation 
to provide a visual representation of the software's architecture. You need to select appropriate ways of describing
the architecture.

You may have noticed that a number of the documents produced by the students at TU Delft use \textsl{views} 
as a way of structuring their description of different aspects of the architecture. The idea of architectural views
is common to different approaches to documenting software architecture. The idea was popularised by Phillipe 
Krutchen in his \href{https://www.cs.ubc.ca/~gregor/teaching/papers/4+1view-architecture.pdf}
{4+1 View Model of Software Architecture}.\footnote{\url{https://www.cs.ubc.ca/~gregor/teaching/papers/4+1view-architecture.pdf}}
You should read this article to understand why different views are often a useful way to describe different
aspects of a software architecture. You do not need to use these views, but you may if you find them useful.

\section{Criteria}

\section{Submission}
The following are \textsl{important} details about how your assignment must be submitted.
Read the following carefully, misreading or misunderstanding the requirements does not except you from them.

\begin{itemize}
    \item The architecture documentation must be written in \href{https://www.markdownguide.org/}{\textsl{markdown}}.\footnote{\url{https://www.markdownguide.org/}}
    \item Submission of the documentation assignment will be via a GitHub repository.\footnote{It is important that you are continually keeping GitHub up to date with your progress.
        Keeping up to date will avoid the merge traffic jam near the due date.}
    \item You will be provisioned a directory in the GitHub repository,
        here you should place your markdown files and any assets (images, code snippets, etc) that are included by the markdown files.
    \item All markdown files in your directory will be joined together in alphabetical order,
        and this will be your final submission.
    \item You can preview your final submission in the \texttt{release} branch.
\end{itemize}

Below is a possible structure of your directory. Note the prefixed numbers to correctly order files.
Alternatively, you can write all content in one file.

\begin{verbatim}
s4435400/
    00-introduction.md
    01-development-view.md
    02-plugin-structure.md
    03-conclusion.md
    assets/
        module-structure.png
        plugin-example.js
\end{verbatim}

\section{Academic Integrity}
As this is a higher-level course, you are expected to be familiar with the importance of academic integrity in general, and the details of UQ's rules.
If you need a reminder, review the \href{https://web.library.uq.edu.au/library-services/it/learnuq-blackboard-help/academic-integrity-modules}
{Academic Integrity Modules}\footnote{\url{https://web.library.uq.edu.au/library-services/it/learnuq-blackboard-help/academic-integrity-modules}}.
Submissions will be checked to ensure that the work submitted is not plagiarised.
If you have quoted or paraphrased any material from another source, it must be correctly \href{https://web.library.uq.edu.au/node/4221/2}
{cited and referenced}\footnote{\url{https://web.library.uq.edu.au/node/4221/2}}.
Use the \href{https://libraryguides.vu.edu.au/ieeereferencing/gettingstarted}
{IEEE referencing style}\footnote{\url{https://libraryguides.vu.edu.au/ieeereferencing/gettingstarted}} for citations and your bibliography.

\end{document}