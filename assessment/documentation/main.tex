\documentclass{csse4400}

\usepackage{CJKutf8}

\hypersetup{
    colorlinks=true,
    linkcolor=violet,
    filecolor=purple,      
    urlcolor=blue,
    citecolor=black,
}

\title{Documenting an Architecture}
\author{Brae Webb \& Richard Thomas}
\date{Semester 1, 2022}

\begin{document}
\maketitle

\section*{Summary}
In this assignment, you will be asked to demonstrate your ability to
\textsl{understand} and subsequently \textsl{communicate} an architecture of an existing software project.
\begin{enumerate}
    \item First, you need to choose a suitable open source software project.
             The project must have non-trivial functionality and architecture.
    \item You will write a report as a 2-4 page markdown document (excluding appendices) which 
             proficiently describes the architecture of your selected software project.
    \item You will then present the key information about the project's
             software architecture to your practical class.
\end{enumerate}

\section{Introduction}
The digital world relies heavily on open source software, as seen by the recent log4j vulnerability.\footnote{https://www.cisa.gov/uscert/apache-log4j-vulnerability-guidance}
Fortunately, open source developers often maintain high quality documentation for the users of their projects.
Unfortunately however, many open source projects don't maintain the same high quality documentation for the architecture of their software projects.
This can cause difficulty for developers who want to contribute to the project, but first need to understand it.

In this project, you have the chance to right this wrong.
Your task is to find an open source software project with a sufficiently complex architecture and document it.
You may optionally choose to share this documentation with the project developers.
You are encouraged to do this, as the perspective of a newcomer to a project is often invaluable to the seasoned developers.

Before looking for projects, read some of the architecture documentation written by students at TU Delft:
\url{https://delftswa.gitbooks.io/desosa2016/content} and \url{https://delftswa.github.io}.

It would also be advantageous to read through one of the architecture descriptions in either volume of
The Architecture of Open Source Applications: \url{http://aosabook.org}.

\section{Finding a Project}
Criteria for the software project:
\begin{itemize}
    \item The project cannot be covered in the tutorials or by the TU Delft students above.
    \item The project must have at least one release within the last year.
\end{itemize}

\noindent%\begin{minipage}{\textwidth}
 Places to look for projects:
\begin{itemize}
    \item GitHub explore page: \url{https://github.com/explore};
    \item Apache project list: \url{https://apache.org/index.html#projects-list};
    \item Awesome Open Source \url{https://awesomeopensource.com/};
    \item in class discussion with other students;
    \item or, ask your tutor.
\end{itemize}
%\end{minipage}

\section{Report Structure}

\begin{description}
    \item[Title] Name of the software project.
    \item[Abstract] Summarise the key points of your document.
    \item[Introduction] Describe the software project, explaining the its key functionality and target users.
    \item[Context] Provide an overview of the software system's context and its external dependencies.
    \item[Architecture] Describe the software's architecture.
    \item[Critique] Analyse the software's architecture, describing its advantages and disadvantages.
    \item[Conclusion] Highlight the key points or lessons learnt about the software's architecture.
\end{description}

\section{Report Content}
How you present the information in your report, is up to you. You will need to select an appropriate notation 
to provide a visual representation of the software's architecture. You need to select appropriate ways of describing
the architecture.

An important aspect of the software system's context is the quality attributes that are important to the success of the project.
You need to identify what you think are the important quality attributes for the project. 
You need to justify why these attributes would be important.
This may require considering how these attributes would be prioritised to make decisions when trade offs need to be made in the design.
Your critique should describe how well the software's architecture supports delivering the quality attributes that you identified as being important.

You may have noticed that a number of the documents produced by the students at TU Delft use \textsl{views} 
as a way of structuring their description of different aspects of the architecture. The idea of architectural views
is common to different approaches to documenting software architecture. The idea was popularised by Phillipe 
Krutchen in his 4+1 View Model of Software Architecture \cite{4+1-model}.
You should read this article to understand why different views are often a useful way to describe different
aspects of a software architecture. You do not need to use these views, but you may if you find them useful.

\section{Presentation}
Presentations will take place in your practical class sessions during weeks 9 to 13.
You will have six minutes for your presentation, plus two minutes for questions.
Your presentation should introduce the software project.
Give an elevator pitch style summary of what problem the project solves and its key features.
Describe which quality attributes you think are most important for the project, and why.
Describe the project's software architecture. Use appropriate views and notation to convey the important aspects of its architecture.
Summarise your critique of the software architecture, highlighting how well it supports delivering the project's key quality attributes.
Your audience is other students in this course. You may assume the audience has knowledge of the course content.

\begin{CJK*}{UTF8}{gbsn}
You are free to structure your presentation however you wish, though you should use some form of slides to support the delivery of information.
One constraint is that your first slide \textbf{must} contain your full name, as recorded in UQ's student enrolment system, and student number.
If you are presenting online, you \textbf{must} show your UQ student card (it does not matter if your UQ student card has expired),
or official government photo id that shows your full name. If your government id does not show your name in Roman characters,
as recorded in UQ's student enrolment system, you need to include a clear image of your government id on your first slide and a textual
representation of your name that can be selected and copied from your slide so that it may be pasted into a translator.
(e.g. If you use your China Resident Identity Card, you must provide clear images of the front and back
of the card. You also need to provide a textual representation of your name in Chinese characters, e.g. 蒙晶.)
\end{CJK*}

\section{Criteria}

\section{Submission}
\subsection{Report Submission}
Your report document is due at \textbf{16:00 (AEST) on March 25, 2022}. Late submissions will \textbf{not} be marked.

The following are \textsl{important} details about how your report must be submitted.
Read the following carefully, misreading or misunderstanding the requirements does not except you from them.

\begin{itemize}
    \item The architecture report must be written in \link{\textsl{markdown}}{https://www.markdownguide.org/}.
    \item Submission of the report component of the assignment will be via a GitHub repository.\footnote{It is important that you are continually keeping GitHub up to date with your progress.
        Keeping up to date will avoid the merge traffic jam near the due date.}
    \item You will be provisioned a directory in the GitHub repository,
        where you should place your markdown files and any assets (images, code snippets, etc) that are included by the markdown files.
    \item All markdown files in your directory will be joined together in alphabetical order,
        and this will be your final submission.
    \item You can preview your final submission in the \texttt{release} branch.
\end{itemize}

\noindent
Below is a possible structure of your directory. Note the prefixed numbers to correctly order files.
Alternatively, you can write all content in one file.

\begin{verbatim}
s4435400/
    00-introduction.md
    01-development-view.md
    02-plugin-structure.md
    03-conclusion.md
    assets/
        module-structure.png
        plugin-example.js
\end{verbatim}

\subsection{Presentation Slides Submission}
Your slides for your presentation are to be submitted to a link provided on BlackBoard.
Your slides are due at \textbf{16:00 (AEST) on April 26, 2022}. A late submission will mean that your presentation will \textbf{not} be marked.


\section{Academic Integrity}
As this is a higher-level course, you are expected to be familiar with the importance of academic integrity in general, and the details of UQ's rules.
If you need a reminder, review the \link{Academic Integrity Modules}{https://web.library.uq.edu.au/library-services/it/learnuq-blackboard-help/academic-integrity-modules}.
Submissions will be checked to ensure that the work submitted is not plagiarised.
If you have quoted or paraphrased any material from another source, it must be correctly \link{cited and referenced}{https://web.library.uq.edu.au/node/4221/2}.
Use the \link{IEEE referencing style}{https://libraryguides.vu.edu.au/ieeereferencing/gettingstarted} for citations and your bibliography.

\bibliographystyle{ieeetr}
\bibliography{articles}

\end{document}