\documentclass{csse4400}

% \teachermodetrue

\usepackage{float}

\usepackage{languages}

\title{Monitoring \& Queues}
\author{Brae Webb}

\date{\week{6}}
\begin{document}

\maketitle

\section{This Week}
This week our goal is to:
\begin{itemize}
  \item investigate the various options to perform health-checks on services;
  \item explore the CloudWatch dashboard to monitor our services; and
  \item deploy an application which processes requests asynchronously using SQS queues.
\end{itemize}


\section{Health-checks}
Health-checks are an way to determine whether or not a service is healthy.
They are a core component to developing reliable and scalable systems.
Health-checks are utilized by systems that manage collections of service instances,
such as Kubernetes, Docker-compose, and of course, AWS Auto-scaling Groups.

In the context of AWS,
health-checks serve help load balancers route traffic only to healthy instances.
Additionally health-checks can instruct auto-scaling groups to spin down unhealthy instances and replace them with new healthy instances.

% There a few methods to determine whether a service is healthy or not.
% \subsection{Liveness checks}

% \subsection{Local health checks}

% \subsection{Dependency health checks}

% \subsection{Anomaly detection}

\section{CloudWatch}

CloudWatch is the AWS solution for monitoring services.
CloudWatch supports service metrics, logging, and alarms.
When working with AWS it is important to understand what CloudWatch can be used for.

\info{
  For the cloud assignment,
  we will be be testing that your submission is able to handle an appropriate load.
  You will be given notice of when this testing will occur.
  The use of CloudWatch to monitor your services and create alarms may give you the ability to manually recover from increased load and perform better in the assignment.
}

\subsection{Metrics}

\subsection{Logging}

\subsection{Alarms}

\section{SQS Queues}


% \bibliographystyle{ieeetr}
% \bibliography{books,ours}

\end{document}