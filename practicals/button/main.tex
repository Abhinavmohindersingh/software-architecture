\documentclass{csse4400}

% \teachermodetrue

\usepackage{float}

\usepackage{languages}

\title{Button - Practice Scalability Assignment}
\author{Evan Hughes \& Brae Webb}

\date{\week{0}}
\begin{document}

\maketitle

% \begin{figure}[h]
%   \href{https://www.oreilly.com/library/view/designing-data-intensive-applications/9781491903063/ch02.html}{
%   }
% \caption{A map of data storage techniques from Designing Data-Intensive Applications \cite{data-intensive}.}
% \end{figure}

\section{This Week}
This week our goal is to:
\begin{itemize}
  \item Build and Deploy the ``Button'' application.
\end{itemize}

\section{Introduction}

\paragraph{Task}

In this example you are working for Richard Thomas's newest business adventure, Button. Button is a company that creates a website where users can push a button to increment a counter. The counter is displayed on the website and updated in real time. Richard saw on your resume that you took Software Architecture and has asked you to design and implement a scalable version of the Button website.

\paragraph{Requirements}

It is expected that the website will be popular and will have interesting peaks of traffic. At the scale that we hope to operate, it would be inappropriate to run the servers required to meet demand at all times. Thus, our service must be elastic --- able to scale up to meet demand and able to scale down to save money when demand is low.

\paragraph{Interface}

The interface specification is available at \url{https://csse6400.uqcloud.net/practicals/buttonspec}.

\paragraph{Implementation}
Please make note of the \link{AWS services}{https://labs.vocareum.com/web/2460291/1564816.0/ASNLIB/public/docs/lang/en-us/README.html\#services} that you can use in the AWS Learner Labs, and the limitations that are placed on the usage of these services. To view this page you need to be logged in to your AWS learner lab environment and have a lab open.

You may \textbf{not} use services or products that run on AWS infrastructure external to your learner lab environment. For example, you may not deploy a third-party product like MonogDB Atlas on AWS and then use it from your service.

\paragraph{Deliverables}
Richard has setup a github account where he would like to keep all the IP of the project. This repository should contain everything required to successfully deploy your application.

\begin{itemize}
  \item Your implementation of the API, including the source code and a mechanism to build the service.
  \footnote{If you use external libraries, ensure that you pin the versions to avoid external changes breaking your application.}
  \item Terraform code that can provision your service in a fresh AWS Learner Lab environment.
  \item A \texttt{deploy.sh} script that can use your terraform code to deploy your application. This script can perform other tasks as required.
\end{itemize}

\textbf{Important Note: 
Ensure your service does not exceed the resource limits of AWS Learner Labs.
For example, AWS will deactivate your account if more than 15 EC2 instances are running.}

\paragraph{Quality Scenario}

Richard is expecting a large influx of users at release so your service must be ready to scale for peak periods of traffic. Richard expects that the service will be able to scale to handle 100 users per second.



\bibliographystyle{ieeetr}
\bibliography{books,ours}

\end{document}
