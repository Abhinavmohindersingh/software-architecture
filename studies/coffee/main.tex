\documentclass{csse4400}

\usepackage{languages}

\title{Coffee Shop}
\author{Brae Webb}

\date{\week{6}}
\begin{document}

\maketitle

\section{Brief}

This week we have introduced the concept of event-driven architectures.
In this tutorial, we will explore and document a project which implements this type of architecture.

\url{https://github.com/CSSE6400/scalable-coffee-shop}

\noindent
The project is an implementation of a coffee shop, implemented by 
\href{https://github.com/sdaschner/scalable-coffee-shop}{Sebastian Daschner}.
The coffee shop exposes a HTTP API, which can be used to stock beans and place drink orders.

Our goal is to run the software locally, begin exploring the implementation, and create C4 diagrams to serve as documentation.

\section{Installation and Running}

For the first 15 minutes of the tutorial,
we will install the application.
To install, we recommend that you use Docker Compose.
The project has a \texttt{docker-compose.yml} configuration in the top-level folder.

\bash{git clone https://github.com/CSSE6400/scalable-coffee-shop^^J
\$ cd scalable-coffee-shop^^J
\$ docker-compose up}

\noindent
Ideally this command will execute without error and start servicing requests as shown in the examples below.

\vspace{2mm}
\info{
The \texttt{docker-compose up} command produces a lot of interleaving output.
If you need to debug why one particular service, e.g. the \texttt{beans} service,
is not working, you can use the \texttt{logs} command to inspect it.

\bash{docker-compose logs -f beans}
}

\vspace{1mm}
\noindent
Add some coffee beans to the beans storage, and confirm they have been created.

\begin{code}[language=bash]{}
$ curl http://localhost:8002/beans/resources/beans -i -XPOST \
    -H 'content-type: application/json' \
    -d '{"beanOrigin": "Colombia", "amount": 10}'
  
  HTTP/1.1 204 No Content
  X-Powered-By: Undertow/1
  Server: WildFly/10
  Date: Fri, 17 Nov 2017 20:49:26 GMT

$ curl http://localhost:8002/beans/resources/beans
{"Colombia":10}
\end{code}

\noindent
Create a new coffee order, and confirm it has been created.

\begin{code}[language=bash]{}
$ curl http://localhost:8001/orders/resources/orders/ -i -XPOST \
    -H 'Content-type: application/json' \
    -d '{"beanOrigin": "Colombia", "coffeeType": "Espresso"}'
  
  HTTP/1.1 202 Accepted
  Connection: keep-alive
  X-Powered-By: Undertow/1
  Server: WildFly/10
  Location: http://localhost:8001/orders/resources/orders/c099c158-b748-4115-bed3-7b5dfff70771
  Content-Length: 0
  Date: Fri, 17 Nov 2017 20:51:16 GMT

$ curl http://localhost:8001/orders/resources/orders/c099c158-b748-4115-bed3-7b5dfff70771
{"status":"started","type":"espresso","beanOrigin":"Colombia"}
\end{code}

\section{Documentation}
In the documentation assignment,
you gained experience creating diagrams of a software architecture.
In the tutorial today,
we will be using those skills to create documentation for the coffee shop application using diagrams.
You can use your favourite diagram software for this task.
For the tutorial, you should produce:

\begin{enumerate}
    \item A context diagram for the system.
    \item A container diagram of the system.
    \item A component diagram for at least the Kafka server.
\end{enumerate}

\noindent
For reference on how event-driven architectures should look in C4, please refer to the Event Driven Architecture handout \cite{events-notes}.

\vspace{2mm}
\info{
    You do not need an in-depth understanding of Kafka to document the architecture.
    For a simple introduction to the concepts of Kafka see the Hortonworks documentation \cite{kafka-basics}.
    For a more in-depth look at how Kafka actually works,
    read the article by Stopford \cite{kafka-in-depth}.
}


\clearpage
\bibliographystyle{ieeetr}
\bibliography{ours,coffee}

\end{document}