\documentclass{csse4400}

% \teachermodetrue

\usepackage{tikz}
\usetikzlibrary{positioning}
\usetikzlibrary{arrows}

\usepackage{float}

\usepackage{enumitem}

\usepackage{languages}

\title{DevOps}
\author{Richard Thomas}

\date{\week[tutorial]{5}}
\begin{document}

\maketitle

\section{Brief}
DevOps is a \link{portmanteau}{https://www.britannica.com/topic/portmanteau-word} of development and operations.
It is intentionally a portmanteau to emphasise that the approach requires a close integration
of development and operations behaviours in one team.
In a proper DevOps environment the development team is responsible for the infrastructure on which their software runs,
not just for the software itself.
This can include considering infrastructure costs in their development budget and estimates.
This week we will
\begin{itemize}
    \item provide an overview of DevOps,
    \item consider what are necessary DevOps practices, and
    \item let you explore how you might implement a DevOps pipeline.
\end{itemize}


\subsection{Introduction to DevOps}
You should be familiar with the concepts of automated testing and continuous integration.
We will now extend that to include continuous testing and deployment,
and then expand on these to provide a full DevOps process.

For an introduction to DevOps read Amazon's description of \link{``What is DevOps?''}
{https://aws.amazon.com/devops/what-is-devops/} \cite{AWS-DevOps}.
Do not worry about the discussion about microservices.
That is an architectural style that will be covered later in the course.
DevOps does not require a microservices architecture,
though there are some benefits of using it.

Skim through the description of implementing DevOps at Wotif\footnote{Now Expedia.} in
\link{``DevOps: Making it Easy to Do the Right Thing''}
{https://search.library.uq.edu.au/permalink/f/tbms52/TN_cdi_webofscience_primary_000383092600012CitationCount}
\cite{CallananMatt2016DMIE}.
Focus on the section titled ``\textbf{Making it Easy}.''
This is where they discuss how they provided the environment to support DevOps.

\subsection{DevOps Practices and Tools}\label{sec:DevOps-Practices}
You should be familiar with some of the practices and tools used in a DevOps process.
Amazon's description of \link{``What is DevOps?''}
{https://aws.amazon.com/devops/what-is-devops/} \cite{AWS-DevOps} lists what they consider to be required practices.
For any practices (aside from microservices) that you are not familiar with, you should follow the links to read a summary of those practices.
Amazon naturally describes their tools for implementing a DevOps pipeline.
You should skim the description of Amazon's tools at \link{``DevOps and AWS''}
{https://aws.amazon.com/devops/} \cite{AWS-DevOps-Tools}.
You do not need to be familiar with these tools, but should have a general idea of what services the tools provide from their summaries.

\newpage
\noindent
Another view of necessary DevOps practices is that it requires continuous
\begin{itemize}[nosep]
    \item development,
    \item integration,
    \item testing,
    \item operations,
    \item deployment,
    \item monitoring, and
    \item feedback.
\end{itemize}
These can only be achieved through automation.

%\subsection{Best Practices}
% Possibly add some commentary about best practices.


\section{Requirements}

\subsection*{Before the Tutorial}
\begin{itemize}
    \item Read the article and web pages indicated above.
    \item Identify a tool that can be used for each of the seven practices listed above.
    \item Come to the tutorial with a list of tools and be prepared to give a ten second summary of each tool.
\end{itemize}

\subsection*{During the Tutorial}
\begin{itemize}
    \item Review and summarise tools that support DevOps practices.
    \item Discuss how different tools can be used to implement a DevOps pipeline.
    \item Define a DevOps pipeline for the Sahara eCommerce case study \cite{service-based-slides}.
\end{itemize}


\section{Outline}

\subsection*{Introduction (5 minutes)}
Introduction to the brief, summarising the idea and value of DevOps and its background.

\teacher{
    Check how many actually have done the readings, and how many have at least skimmed parts of some of the readings.
}

\subsection*{Small Group Discussion (12 minutes)}
In small groups, describe the tools you identified that support each of the DevOps practices.
Briefly explain how they can be integrated to deliver a complete DevOps pipeline.

\teacher{
    This could be done by having students go around their table and describe the tools they identified to everyone else at their table.
    Assuming a group of students have not come prepared with a list of tools,
    challenge them to spend 5 minutes finding possible tools.
}

\subsection*{Class Discussion (8 minutes)}
With the class, summarise the tools identified by each group and which DevOps practices they support.
Consider the following questions:
\begin{itemize}
    \item How well does each tool support one or more of the DevOps practices
          (i.e. development, integration, testing, operations, deployment, monitoring, and feedback)?
    \item Are there advantages to having tools support multiple practices,
          or is it easier to integrate tools if they only support a single practice?
\end{itemize}

\teacher{
    Depending on timing, you may need to constrain discussion in this section or skip it all together.
}

\subsection*{Pipeline Design (15 minutes)}
In a small group, design a DevOps pipeline for the Sahara eCommerce case study \cite{service-based-slides}.
Use the service-based architecture approach for the project.
\begin{itemize}
    \item What types of tools would be required?
    \item Which specific tools would you choose?
    \item On which type of computing infrastructure would you deliver the system?
    \item What parts of the deployment and operations processes could be automated?
\end{itemize}

\subsection*{Pipeline Discussion (10 minutes)}
With the class, present a few of the pipelines summarising how the tools support delivering an integrated DevOps process.
Consider the following questions:
\begin{itemize}
    \item How well does the entire tool chain support all seven DevOps practices
          (i.e. development, integration, testing, operations, deployment, monitoring, and feedback)?
    \item How well would the tool chain support an integrated perception of DevOps as an organisational process?
    \item Would additional tooling (e.g. scripting) be required to enable smooth integration of the tools?
          Where would that be required?
\end{itemize}


\section{Challenges}

\subsection*{Challenge 1: DevOps in Practice}
Read or skim \link{``DevOps Capabilities, Practices, and Challenges: Insights from a Case Study''}
{https://search.library.uq.edu.au/permalink/f/tbms52/TN_cdi_arxiv_primary_1907_10201}
\cite{SenapathiMali2018DCPa}.
Consider the differences between the capabilities and technological enablers mentioned in the article,
and the seven DevOps practices listed in section \ref{sec:DevOps-Practices}.
\begin{itemize}
    \item Do the seven necessary DevOps practices map perfectly to the enablers in the article by Senapathi et al \cite{SenapathiMali2018DCPa}?
    \item If there are some mismatches, are they important enough that they should be a required practice?
\end{itemize}

\noindent
Post your thoughts about these questions in the \textbf{pracs-tut} channel on Slack.

\subsection*{Challenge 2: DevOps Best Practice}
Scan the literature\footnote{This may be formal peer-reviewed sources or published websites or blogs.}
and identify what is currently considered to be DevOps ``best practice.''
Summarise your findings and post these in the \textbf{pracs-tut} channel on Slack.
include links to the sources you used in coming to your conclusion.

\subsection*{Challenge 3: Reuse or Create}
Based on your reading and experience, is it better to use an existing DevOps process and pipeline,
or is it better for a team to define their own process and create their own pipeline?
Pick one of the sides of this point and post a paragraph arguing your view in the \textbf{pracs-tut} channel on Slack.


\bibliographystyle{ieeetr}
\bibliography{articles,ours}

\end{document}