\documentclass{slide}

\title{Architectural Views}
\subtitle{CSSE6400}
\author{Richard Thomas}
\date{\week{2}}

\usepackage{languages}
\usepackage{changepage}

\usepackage{tikz}
\usetikzlibrary{positioning}
\usetikzlibrary{fit}
\usetikzlibrary{arrows.meta}

\begin{document}

\maketitle

\point[\textit{Interesting} Software is Complex]{Many aspects to the design of its architecture.}

\point[Architectural Design]{Managing technical complexity.}

\questionanswer{How do you describe a complex architecture, without making it too difficult to understand?}
{\highlight{Architectural Views}\\~~~~ -- Only consider one aspect at a time.}


\begin{frame}{Architectural Views}

\Large{
\begin{itemize}
    \item<1-> 4+1 Views \cite{4+1-model}
    \begin{itemize}
        \large{\item[$\bullet$] logical, process, development, physical, scenario}
    \end{itemize}
    \item<2-> Software Architecture in Practice \cite{bass2021software}
    \begin{itemize}
        \large{\item[$\bullet$] module, component-and-connector, allocation}
    \end{itemize}
    \item<3-> Rozanski and Woods \cite{rozanski}
    \begin{itemize}
        \large{\item[$\bullet$] context, building block, runtime, deployment}
    \end{itemize}
    \item<4-> NATO Architecture Framework \cite{nafv4}
    \begin{itemize}
        \large{\item[$\bullet$] concepts, service, logical, physical resource, architecture foundation}
    \end{itemize}
    \item<5-> The Open Group Architecture Framework (TOGAF) \cite{togaf}
    \item<5-> ISO/IEC/IEEE 42010:2011 \cite{iso42010}
\end{itemize}
}

\end{frame}


\begin{frame}{4+1 Views}

\Large{
\begin{description}
    \item<1->[Logical] -- \textit{Structure} of how the software is implemented.
    \begin{itemize}
        \large{\item[$\bullet$] components/classes, relationships, interactions}
    \end{itemize}
    \item<2->[Process] -- \textit{Dynamic} behaviour.
    \begin{itemize}
        \large{\item[$\bullet$] concurrency \& distribution, fault tolerance, process control, ...}
    \end{itemize}
    \item<3->[Development] -- \textit{Organisation} of the software in the development environment.
    \item<4->[Physical] -- \textit{Map} executable software containers to hardware.
    \begin{itemize}
        \large{\item[$\bullet$] address non-functional requirements}
        \begin{itemize}
            \item[$\bullet$] availability, reliability, scalability, throughput, ...
        \end{itemize}
    \end{itemize}
    \item<5->[Scenario] -- \textit{Demonstrate} functionality delivered by architecture.
    \begin{itemize}
        \large{\item[$\bullet$] use case details}
        \begin{itemize}
            \item[$\bullet$] \textit{drive} functional design of architecture
            \item[$\bullet$] \textit{validate} design of architecture
            \item[$\bullet$] \textit{illustrate} purpose of architecture
        \end{itemize}
    \end{itemize}
\end{description}
}

\end{frame}


\begin{frame}{Diagrams \& Notation}

\Large{
\begin{itemize}
    \item<1-> A \highlight{good} diagram is worth a thousand words.
    \begin{itemize}
        \large{\item A thousand diagrams is just confusing.}
    \end{itemize}
    \vspace{5mm}
    \item<2-> UML -- formal, well-defined language \cite{uml}
    \item<2-> C4 -- informal, simple structure \cite{brown2022c4}
    \vspace{2mm}
    \item<2-> You probably don't want to know about alternatives.
\end{itemize}
}

\end{frame}


\point[Reading...]{``Architectural Views'' Notes\footnote{Remember, I said you had to read the notes.} \cite{view-notes}}


\references{articles,books,ours}

\end{document}