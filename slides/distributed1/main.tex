\documentclass{slide}
\usepackage{tikz}

\usetikzlibrary{arrows}

\usepackage{tabto}

\usepackage{languages}

% \usepackage{pgfpages}
% \setbeameroption{show notes on second screen}

\title{Distributed Systems I}
\subtitle{Software Architecture}
\author{Brae Webb}
\date{\week{5}}

\titlegraphic {
    \tweet%
    {images/mathiasverraes}%
    {Mathias Verras}%
    {mathiasverraes}%
    {There are only two hard problems in distributed systems:\\  2. Exactly-once delivery\\ 1. Guaranteed order of messages\\ 2. Exactly-once delivery}%
    {https://twitter.com/mathiasverraes/status/632260618599403520}%
}

\begin{document}

\maketitle

\note{Lecture Goal: Balance a healthy love-hate relationship with distributed systems}

\point[Going forward]{
    Investigating architectures that are \highlight{distributed}.
}

\point[Distributed Systems Series]{
    \begin{description}
        \item[Distributed I] \highlight{Reliability} and \highlight{scalability} of \highlight{stateless} systems.
        \item[Distributed II] \highlight{Complexities} of \highlight{stateful} systems.
        \item[Distributed III] \highlight{Hard problems} in distributed systems.
    \end{description}
}

\point[What are the benefits?]{
    \begin{itemize}
        \item Improved \highlight{reliability}.
        \item Improved \highlight{scalability}.
        \item Improved \highlight{latency}.
    \end{itemize}
}

\note{Some systems are inherently distribued.}

\point[What are the drawbacks?]{
    \begin{itemize}
        \item Increased \highlight{complexity}.
        \item Increased \highlight{attack vector}.
        \item Increased \highlight{latency}.
        \item Introduce \highlight{consistency} problems.
    \end{itemize}
}

\note{We'll look at a few reasons that distributed systems are \highlight{fundamentally} quite challenging}

\section{Fallacies}

\point[A \textsl{few} reasons for complexity]{
    The Fallacies of \highlight{Distributed Computing}.
}

\note{Sun Microsystems in 1994, primarily accredited to Peter Deutsch (doy-ch)}

\point[Fallacy \#1]{The network is reliable.}

\image{diagrams/Success}

\image{diagrams/SendDrop}

\image{diagrams/SendDropResend}

\note{Solve it by resending it}

\image{diagrams/SendDropDos}

\note{If the service goes down and all clients are re-trying, the service is in for a shock when it comes back, we solve this with \highlight{exponential backoff}}

\begin{frame}[fragile]{Exponential backoff}
\begin{code}[language=python,morekeywords={do}]{}
retry = True
do:
    status = service.request()

    if status != SUCCESS:
        wait(2 ** retries)
    else:
        retry = False
while (retry and retries < MAX_RETIRES)
\end{code}
\end{frame}

\image{diagrams/ReceiveDrop}

\image{diagrams/ReceiveDropResend}

\note{Causes duplicate actions, problem for ordering/payments}

\image{diagrams/ReceiveDropToken}

\note{Use tokens to prevent duplicates.}

\point[Fallacy \#2]{Latency is zero.}

\point[Network Statistics]{
    \begin{description}[leftmargin=!,labelwidth=\widthof{Data}]
        \item[Home to UQ] \only<2->{20.025ms}
        \item[Home to us-east-1] \only<3->{249.296ms}
        \item[EC2 to EC2] \only<4->{0.662ms}
    \end{description}
}

\note{Be mindful when designing distributed systems. Network call much slower then local call.}

\point[Fallacy \#3]{Bandwidth is infinite.}

\note{Similar to previous fallacy, be mindful, distributed calls clog up network.}

\definition{Stamp Coupling}{
    Components which share a composite data structure.
}

\point[Fallacy \#4]{The network is secure.}

\image[height=1.07\textheight]{diagrams/Sahara}

\note{Authentication only occurs when entering Sahara data centre}

\image[height=1.1\textheight]{diagrams/NetworkInjection}

\note{Bad actor gets access via one insecure node, network is compromised. Practice defence in depth.}

\point[Fallacy \#5]{The topology never changes.}

\note{Topology changes all the time, cloud has just made this easier. Don't rely on static IPs. Don't assume consistent latency.}

\point[Fallacy \#6]{There is only one administrator.}

\note{Things spontaneously break. Who can help you?}

\point[Fallacy \#7]{Transport cost is zero.}

\point[Remember]{
Distributed systems are \highlight{hard}.

The choice to use them should be \highlight{well considered}.
}

\note{Can often introduce more problems than they solve}

\point[When you need to, maybe prove it?]{
    \centering \youtubevideo{images/isabelle-thumb}{https://youtube.com/watch?v=7w4KC6i9Yac}
}

\point[Or, more realistically,]{
    Use existing algorithms and software.
}

\point[Distributed Systems Series]{
    \begin{description}
        \item[Distributed I] \highlight{Reliability} and \highlight{scalability} of \highlight{stateless} systems.
            \color{gray}{
        \item[Distributed II] Complexities of stateful systems.
        \item[Distributed III] Hard problems in distributed systems.}
    \end{description}
}

\point[Stateless vs. Stateful Systems]{
    \begin{description}
        \item[Stateless] Does \highlight{not} utilize \highlight{persistent data}.
        \item[Stateful] Does utilize \highlight{persistent data}.
    \end{description}
}

\question{What makes software \highlight{reliable}?}

\definition{Reliable Software}{Continues to work, even when things go wrong.}

\definition{Fault}{Something goes wrong.}

\quote[Howard and LeBlanc]{Death, taxes, and computer system failure are all inevitable to some degree.\\\highlight{Plan for the event.}}

\point[Reliable software is]{Fault \highlight{tolerant}.}

\note{John von Neumann built fault tolerant hardware in the 1950s.}

\point[Problem]{Individual computers fail \highlight{all the time}}

\note{10-50 years hard-drive lifetime. 10,000 disks will fail daily. Google last had 2.5 million servers.}

\point[Solution]{Spread the risk of faults over \highlight{multiple computers}, or, \highlight{nodes}.}

\point[Spreading Risk]{
    \Large
    If you have software that works with \highlight{just one} computer,
    spreading the software over \highlight{two} computers \highlight{halves} the risk that your software will fail.
    \\[2em]
    \only<2->{Adding \highlight{100} computers reduces the cuts the risk by \highlight{100}.}
}

\image[height=1.07\textheight]{diagrams/Sahara}

\note[itemize]{
    \item Why is this software somewhat reliable?
    \item Any individual service can go down and the rest still work.
    \item Can we do better?
    \item Can a service go down but have that service still work?
}

\question{Who has used \highlight{auto-scaling}?}

\point[Auto-scaling terminology]{
\Large
\begin{description}[<+->]
    \item[Auto-scaling group] A \highlight{collection of instances} managed by auto-scaling.
    \item[Capacity] Amount of instances \highlight{currently} in an auto-scaling group.
    \item[Desired Capacity] Amount of instances \highlight{we want to have} in an auto-scaling group.
    \item[Scaling Policy] How to determine the desired capacity.
    \item[Minimum/Maximum Capacity] \highlight{Hard limits} on the minimal and maximum amount of instances.
\end{description}
}

\point[What we really want]{
    {\color{secondary}Desired Capacity} Amount of \highlight{healthy} instances we want to have in an auto-scaling group.
}

\point[Health check]{
    Mechanism to determine whether an instance is \highlight{healthy}.
}

\point[Auto-scaling]{An example}

\image[height=1.07\textheight]{diagrams/Sahara}

\note{Product service keeps going down}

\image{diagrams/ProductIsolate}

\note{We might expect product service to have a much higher load than other services}

\image{diagrams/ProductEC2}

\image{diagrams/ProductScale}

\note{Use an auto-scaling group to replicate the service}

\image{diagrams/ProductScaleReal}

\note{What's the problem?}

\image[width=\textheight]{diagrams/ProductScaleLB}

\note{Traffic was all sent through the one instance, load balancer routes to all}

\image[width=\textheight]{diagrams/SaharaScaled}

\begin{frame}{In Summary}
\Large
\begin{description}
    \item[Simplicity] \only<2->{\highlight{Minimal network communication} (compared to other distributed systems), less impacted by fallacies.}
    \item[Reliability] \only<3->{Traffic is spread to various services, still \highlight{partially operational} if one goes down. Auto-scaling allows for \highlight{basic replication}.}
    \item[Scalability] \only<4->{Auto-scaling and load balancing allows \highlight{individual services to scale}. However, the \highlight{database is a bottle-neck}.}  
\end{description}
\end{frame}

\note{\highlight{database is a bottle-neck} is foreshadowing}

% \definition{Distributed Computing}{Multiple software components that are on multiple computers, but run as a single system}

% \point[The Problem]{
%     \begin{center}
%     \begin{tikzpicture}
%         \node[circle,draw, minimum size=1cm] (A) at  (0,0) {Distributed};
%         \node[circle,draw, minimum size=1cm] (B) at  (8,0)  {Simplicity};
%         \draw[<->] (A) -- (B);
%     \end{tikzpicture}
%     \end{center}
% }

% \point{A lot of modern software development focuses on dealing with the \highlight{complexity} of distributed systems.}

% \question{What makes distributed computing complex?}

% \point{
%     \begin{itemize}
%         \item Faults
%         \item Asynchronous communication
%         \item Monitoring
%         \item And much more\dots
%     \end{itemize}
% }


% \references{articles,books}

\end{document}
