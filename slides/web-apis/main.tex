\documentclass{slide}
% \usepackage{pgfpages}

% \setbeameroption{show notes on second screen}

\usepackage{tikz}
\usetikzlibrary{backgrounds}
\usetikzlibrary{shapes}
\usetikzlibrary{tikzmark}
\usetikzlibrary{calc}
\usetikzlibrary{positioning}
\usetikzlibrary{arrows}
\usetikzlibrary{fit}

\title{Web APIs}
\subtitle{CSSE6400}
\author{Brae Webb}
\date{\week{8}}

\usepackage{languages}

\newcommand{\component}[3]{%
\tikzmarknode{#1}{\colorbox{#2}{\color{white}#3}}%
}

\definecolor{navyblue}{rgb}{0.0, 0.4, 0.65}
\definecolor{neonfuchsia}{rgb}{1.0, 0.25, 0.39}
\definecolor{mediumseagreen}{rgb}{0.24, 0.7, 0.44}
\definecolor{mediumorchid}{rgb}{0.73, 0.33, 0.83}
\definecolor{meatbrown}{rgb}{0.9, 0.72, 0.23}

\begin{document}

\maketitle

\point[Goals]{
    \begin{itemize}[<+->]
        \item Review existing networking knowledge.
        \item Understand \highlight{URLs}.
        \item Understand \highlight{HTTP} protocol and methods.
        \item Understand \highlight{RESTful} APIs.
        \item \highlight{Build} a basic RESTful API.
    \end{itemize}
}

\begin{frame}{OSI Model}
\centering
\begin{tikzpicture}[scale=0.5]

    \draw[very thick,draw=secondary,top color=white] (-3,0) rectangle (3,-1.5);
    \draw[very thick,draw=secondary,top color=white] (-3.5,-2) rectangle (3.5,-3.5);
    \draw[very thick,draw=secondary,top color=white] (-4,-4) rectangle (4,-5.5);
    \draw[very thick,draw=secondary,top color=white] (-4.5,-6) rectangle (4.5,-7.5);
    \draw[very thick,draw=secondary,top color=white] (-5,-8) rectangle (5,-9.5);
    \draw[very thick,draw=secondary,top color=white] (-5.5,-10) rectangle (5.5,-11.5);
    \draw[very thick,draw=secondary,top color=white] (-6,-12) rectangle (6,-13.5);

    \node at (0, -0.75) (app) {Application Layer};
    \node [below of=app] (pres) {Presentation Layer};
    \node [below of=pres] (sess) {Session Layer};
    \node [below of=sess] (trans) {Transport Layer};
    \node [below of=trans] (net) {Network Layer};
    \node [below of=net] (data) {Data Link Layer};
    \node [below of=data] {Physical Layer};
\end{tikzpicture}
\end{frame}

\begin{frame}{OSI Model}
\centering
\begin{tikzpicture}[scale=0.5]

    \draw[very thick,draw=black!10,top color=white] (-3,0) rectangle (3,-1.5);
    \draw[very thick,draw=black!10,top color=white] (-3.5,-2) rectangle (3.5,-3.5);
    \draw[very thick,draw=black!10,top color=white] (-4,-4) rectangle (4,-5.5);
    \draw[very thick,draw=secondary,top color=white] (-4.5,-6) rectangle (4.5,-7.5);
    \draw[very thick,draw=black!10,top color=white] (-5,-8) rectangle (5,-9.5);
    \draw[very thick,draw=black!10,top color=white] (-5.5,-10) rectangle (5.5,-11.5);
    \draw[very thick,draw=black!10,top color=white] (-6,-12) rectangle (6,-13.5);

    \node [color=black!30] at (0, -0.75) (app) {Application Layer};
    \node [color=black!30,below of=app] (pres) {Presentation Layer};
    \node [color=black!30,below of=pres] (sess) {Session Layer};
    \node [below of=sess] (trans) {Transport Layer};
    \node [color=black!30,below of=trans] (net) {Network Layer};
    \node [color=black!30,below of=net] (data) {Data Link Layer};
    \node [color=black!30,below of=data] {Physical Layer};

    \only<2>{\node [right=1cm of trans] {TCP/UDP (CSSE2310)};}
\end{tikzpicture}
\end{frame}

\point[TCP/UDP]{
    Low-level with \highlight{minimal abstraction}.
}

\point[TCP/UDP]{
    \highlight{Impractical} for building web APIs.
}

\begin{frame}{OSI Model}
\centering
\begin{tikzpicture}[scale=0.5]

    \draw[very thick,draw=secondary,top color=white] (-3,0) rectangle (3,-1.5);
    \draw[very thick,draw=black!10,top color=white] (-3.5,-2) rectangle (3.5,-3.5);
    \draw[very thick,draw=black!10,top color=white] (-4,-4) rectangle (4,-5.5);
    \draw[very thick,draw=black!10,top color=white] (-4.5,-6) rectangle (4.5,-7.5);
    \draw[very thick,draw=black!10,top color=white] (-5,-8) rectangle (5,-9.5);
    \draw[very thick,draw=black!10,top color=white] (-5.5,-10) rectangle (5.5,-11.5);
    \draw[very thick,draw=black!10,top color=white] (-6,-12) rectangle (6,-13.5);

    \node at (0, -0.75) (app) {Application Layer};
    \node [color=black!30,below of=app] (pres) {Presentation Layer};
    \node [color=black!30,below of=pres] (sess) {Session Layer};
    \node [color=black!30,below of=sess] (trans) {Transport Layer};
    \node [color=black!30,below of=trans] (net) {Network Layer};
    \node [color=black!30,below of=net] (data) {Data Link Layer};
    \node [color=black!30,below of=data] {Physical Layer};

    \only<2>{\node [right=1cm of app] {HTTP/HTTPS (CSSE6400)};}
\end{tikzpicture}
\end{frame}

\point[The anatomy of]{URLs}

\begin{frame}[fragile]
\huge
\ttfamily
\begin{center}
\component{proto}{mediumseagreen}{http}%
://%
\component{host}{navyblue}{example.com}%
/%
\component{path}{neonfuchsia}{hello-world}%
\end{center}
\begin{tikzpicture}[overlay,remember picture,>=stealth,nodes={align=left,inner ysep=1pt},<-]
    % For 'proto'
    \path (proto.south) ++ (0,-1.5em) node[anchor=north west,color=mediumseagreen!87] (proto-lab){$Protocol$};
    \draw [color=mediumseagreen!87](proto.south) |- ([xshift=-0.3ex,color=blue]proto-lab.south east);

    % For 'host'
    \path (host.north) ++ (0,2em) node[anchor=south east,color=navyblue!87] (scalep){$Host name$};
    \draw [color=navyblue!87](host.north) |- ([xshift=-0.3ex,color=red]scalep.south west);

    % For 'path'
    \path (path.south) ++ (0,-1.5em) node[anchor=north west,color=neonfuchsia!87] (mean){$Path$};
    \draw [color=neonfuchsia!87](path.south) |- ([xshift=-0.3ex,color=blue]mean.south east);
\end{tikzpicture}
\end{frame}


\begin{frame}[fragile]
\huge
\ttfamily
\begin{center}
\component{proto}{mediumseagreen}{http}%
://%
\component{host}{navyblue}{example.com}%
:\component{port}{mediumorchid}{80}%
/%
\component{path}{neonfuchsia}{hello-world}%
\end{center}
\begin{tikzpicture}[overlay,remember picture,>=stealth,nodes={align=left,inner ysep=1pt},<-]
    % For 'proto'
    \path (proto.south) ++ (0,-1.5em) node[anchor=north west,color=mediumseagreen!87] (proto-lab){$Protocol$};
    \draw [color=mediumseagreen!87](proto.south) |- ([xshift=-0.3ex,color=blue]proto-lab.south east);

    % For 'host'
    \path (host.north) ++ (0,2em) node[anchor=south east,color=navyblue!87] (scalep){$Host name$};
    \draw [color=navyblue!87](host.north) |- ([xshift=-0.3ex,color=red]scalep.south west);

    % For 'path'
    \path (path.south) ++ (0,-1.5em) node[anchor=north west,color=neonfuchsia!87] (mean){$Path$};
    \draw [color=neonfuchsia!87](path.south) |- ([xshift=-0.3ex,color=blue]mean.south east);

    % For 'port'
    \path (port.north) ++ (0,1em) node[anchor=south west,color=mediumorchid!87] (mean){$Port$};
    \draw [color=mediumorchid!87](port.north) |- ([xshift=-0.3ex,color=blue]mean.south east);
\end{tikzpicture}
\end{frame}


\begin{frame}[fragile]
\Large
\ttfamily
\begin{center}
\component{proto}{mediumseagreen}{http}%
://%
\component{host}{navyblue}{example.com}%
/%
\component{path}{neonfuchsia}{hello-world}%
?%
\component{query}{meatbrown}{key=value}%
\end{center}
\begin{tikzpicture}[overlay,remember picture,>=stealth,nodes={align=left,inner ysep=1pt},<-]
    % For 'proto'
    \path (proto.south) ++ (0,-1.5em) node[anchor=north west,color=mediumseagreen!87] (proto-lab){$Protocol$};
    \draw [color=mediumseagreen!87](proto.south) |- ([xshift=-0.3ex,color=blue]proto-lab.south east);

    % For 'host'
    \path (host.north) ++ (0,2em) node[anchor=south east,color=navyblue!87] (scalep){$Host name$};
    \draw [color=navyblue!87](host.north) |- ([xshift=-0.3ex,color=red]scalep.south west);

    % For 'path'
    \path (path.south) ++ (0,-1.5em) node[anchor=north west,color=neonfuchsia!87] (mean){$Path$};
    \draw [color=neonfuchsia!87](path.south) |- ([xshift=-0.3ex,color=blue]mean.south east);

    % For 'query'
    \path (query.north) ++ (0,2em) node[anchor=south east,color=meatbrown!87] (scalep){$Query\,Variable$};
    \draw [color=meatbrown!87](query.north) |- ([xshift=-0.3ex,color=red]scalep.south west);
\end{tikzpicture}
\end{frame}

\point[HTTP]{
    A \highlight{request-response} abstraction for networking.
}

\point[HTTP Request]{
    \begin{description}[width=URL]
        \item[URL] An endpoint to send request to.
        \item[Method] Described later.
        \item[Headers] Specify type of data, e.g. JSON, HTML, etc.
        \item[Body] Optional extra data to include.
    \end{description}
}

\point[HTTP Response]{
    \begin{description}[width=Body]
        \item[Status Code] A number between 100 and 599 giving details about the response.
        \item[Headers] Specify type of response data, e.g. JSON, HTML, etc.
        \item[Body] Content of the response.
    \end{description}
}

\point[Status Codes]{
    \begin{description}
        \item[200s] Indicate the request was \highlight{successful}, 200 is most common.
        \item[300s] \highlight{Redirects} the client to another location.
        \item[400s] Indicates that the \highlight{request was wrong}\\
        {\small e.g. 404 meaning that the request was for something that doesn't exist.}
        \item[500s] Indicates that the \highlight{server had a problem} fulfilling the request.
    \end{description}
}

\point[Types of HTTP communication]{HTTP Methods}

\begin{frame}{HTTP Methods}
\Huge
\begin{description}[<+->][width=GET]
    \item[GET] \highlight{Query} for information.
    \item[POST] \highlight{Create} resource.
    \item[PUT] \highlight{Update} resource.
    \item[DELETE] \highlight{Delete} resource.  
\end{description}
\end{frame}

\point{Examples}

\begin{frame}[fragile]
\begin{code}[language=python]{app.py}
from flask import Flask

app = Flask(__name__)

@app.route("/")
def hello_world():
    return "Hello, World!"

if __name__ == "__main__":
    app.run(port=6400)
\end{code}
\end{frame}

\point[Result]{
\vspace{10pt}
\browser{http://localhost:6400/}{http://localhost:6400/}{
    \hspace{10pt}\tiny Hello, World!
}
}

\begin{frame}[fragile]
\begin{code}[language=javascript]{app.js}
const express = require('express')
const app = express()
const port = 6400

app.get('/', (req, res) => {
    res.send('Hello, World!')
})

app.listen(port, () => {
    console.log(`Example app listening on port ${port}`)
})
\end{code}
\end{frame}



\begin{frame}[fragile]
\begin{code}[language=python]{app.py}
from flask import Flask

app = Flask(__name__)

@app.route("/health")
def hello_world():
    return {"status": "okay!"}

if __name__ == "__main__":
    app.run(port=6400)
\end{code}
\end{frame}

\point[Result]{
\vspace{10pt}
\browser{http://localhost:6400/}{http://localhost:6400/}{
    \hspace{10pt}\tiny \texttt{\{"status": "okay!"\}}
}
}

\begin{frame}[fragile]
\begin{code}[language=javascript]{app.js}
const express = require('express')
const app = express()
const port = 6400

app.get('/', (req, res) => {
    res.send({"status": "okay!"})
})

app.listen(port, () => {
    console.log(`Example app listening on port ${port}`)
})
\end{code}
\end{frame}



\begin{frame}[fragile]
\begin{code}[language=python]{app.py}
from flask import Flask
from flask import request

app = Flask(__name__)

@app.route("/echo", methods=["POST"])
def hello_world():
    return request.json.say

if __name__ == "__main__":
    app.run(port=6400)
\end{code}
\end{frame}

\begin{frame}[fragile]
\begin{code}[language=bash]{}
>>> curl -X POST \
-H "Accept: application/json" \
-H "Content-Type: application/json" \
"http://localhost:6400" \
-d '{
    "say" : "Hello, World",
}'
Hello, World
\end{code}
\end{frame}

\begin{frame}[fragile]
\begin{code}[language=javascript]{app.js}
const express = require('express')
const app = express()
const port = 6400

app.post('/', express.json(), (req, res) => {
    res.send(req.body.say)
})

app.listen(port, () => {
    console.log(`Example app listening on port ${port}`)
})
\end{code}
\end{frame}

% \references{articles,books}

\end{document}