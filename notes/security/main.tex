\documentclass{csse4400}

\title{Security Principles}
\author{Brae Webb}

\date{January 20, 2022}

\begin{document}

\makecover

\maketitle

\section{Introduction}

One quality attributes that developers often overlook is security.
Security can be the cause of a great deal of frustration for developers;
there is no comfortable architecture, nor command-line tool to magically make an application secure.
As the world depends on technology more than ever while cyber attacks become more devastating,
it has become crystal clear that security is everyones responsibility.
As users of technology, as developers, and architects, we all need to ensure we take security seriously.

Learning, and for that matter, teaching, how to make software secure is no easy task.
Every application has different flaws and risks, every attack vector and exploit is unique; managing to keep up with it all is daunting.
None of us will ever be able to build a completely secured system but that is not a reason to stop trying.
As developers and architects, security should be an on-going process in the back of your minds.
A nagging voice constantly asking `what if?'.

We introduce security first to set the example.
As we go through this course, the principle of security will recur again and again.
With each architecture introduced, we will stop and ask ourselves `what if?'.
In your future careers, you should endeavour to carry this same practice.
Each feature, pipeline, access control change, or code review, ask yourself, `what are the security implications?'.

With that said, there are some useful principles, and a handful of generic best practices which we will cover.
Even if you follow the best practices and embody the principles,
you will still be hopeless insecure, unless, you constantly reflect on the security implications of your every decision.

\section{The Self}
Before we even fantasize about keeping our applications secure, let's review if you're secure right now.
As developers we often have heightened privilege and access, at times above that of even the CEOs.
If you aren't secure, nor is anything you work on.
Let's review some of the basics.

\textbf{Keep your software up to date.}
Are you running the latest version of your operating system?
The latest chrome, or firefox, or god-forbid, internet explorer?
If not then there is a good chance you are currently at risk.
Software updates, while annoying, provide vital patches to discovered exploits,
you need to keep your software up to date.

\textbf{Use multi-factor authentication.}
This should be hard to explain to a grandmother but this should be obvious to software developers.
One million passwords are stolen every week \cite{password-security}.
If you don't have some sort of multi-factor authentication enabled, hackers can access your account immediantly after discovering your password.

\textbf{Be smart.}
The final way that you can secure yourself is to be smart.
As a developer, you should be perfectly capable of keeping yourself secure online.
Be wary of phishing and running untrusted software.
You know how the online world works!


\section{Practices of Secure Software}


\section{Principles of Securing Software}


\begin{drafting}
\begin{itemize}
    \item introduce security
    \item security is an on-going process
    \item best kept in the back of your head
    \item you will never be completely secure, but that shouldn't stop you trying
    \item couple of useful principles
    \begin{itemize}
        \item principle of least privilege
        \item principle of separation of duties
        \item principle of defense in depth
        \item principle of failing securely (facebook example)
        \item principle of open design (2310)
        \item principle of avoiding security by obscurity
        \item principle of minimizing attack surface area
    \end{itemize}
    \item couple of best practices
    \begin{itemize}
        \item sanitizing inputs
        \item use encryption where reasonable
        \item keep software up to date
        \item use 2FA
        \item use firewalls
        \item track your systems
        \item use a password manager
        \item be smart, trust but verify
    \end{itemize}
    \item when in doubt, hire an expert
\end{itemize}
\end{drafting}


\bibliographystyle{plain}
\bibliography{articles}


\end{document}