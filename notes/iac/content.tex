\title{Infrastructure as Code}
\maketitle

\todo{I'm not happy with this introduction}

\section{Introduction}
Configuration management can be one of the more challenging aspects of software development.
Configuration in terms of configuration management is given a wide scope.
Configuration refers to software dependencies, operating system configuration, and every aspect of a machines environment.
Locally developers have partially overcome this challenge.
The configuration isolation afforded by containerization enables developers to fully specify the required configuration.

When we consider configuration management of our production environments and other infrastructure,
there is more which we need to consider.
Containers allow us to specify the configuration of a single machine.
However, we really need the ability to specify our entire infrastructures configuration.
How many machines do we have?
How do they communicate over the network?
What software are they running?
How do they support an increased load?
These are some of the questions which infrastructure as code seeks to answer.

Infrastructure as Code (IaC) originates from virtualization.
In the `Iron Age' of computing, installing a new server was rate limited by how quickly you could shop for and order a new machine.
Once the machine arrived at your company, some number of weeks after the purchase,
someone was tasked with setting it up, configuring it to talk to other machines, installing the software it needed.
Compared to the weeks it takes to aquire the machine, a day of configuration is not so bad.
With virtualization one physical machine can be the home of numerous virtual machines.
Each one of these machines requires configuration.
Now, with new machines available within minutes a day of configuring is out of the question.
And so we have infrastructure as code.

Infrastructure as code can be simple.
A simple shell script which installs dependencies is infrastructure as code.
Infrastructure as code is a specification of our ideal world.
IaC can be imperative but more often we find it useful to be declarative.
In declarative IaC we specify our idealized world and our compiler shapes the real world to match.

Sometimes we can aptly a concept in such a way that makes the need for it clear.
Here I will ask for your trust.
I'll promise you that IaC is useful,
then we'll learn how to wield it properly and hope that the benefits become clear.
