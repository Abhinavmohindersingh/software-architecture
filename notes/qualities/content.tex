\title{Quality Attributes}
\maketitle

\section{Introduction}
Software projects are defined by their functional and non-functional requirements.
The functional requirements specify what the software should do while the
non-functional requirements specify properties required for the project to be successful.
Non-functional requirements are called Quality Attributes.
Often quality attributes are expressed with terms ending in -ility.
Medical software needs reliability. Social media needs availability.
Census software needs scalability.

Quality attributes are one of the main focuses of a software architect.
The architecture of software is the key factor which influences quality attributes.
Achitects are responsible for identifying the important attributes for their project
and implementing achitecture and practices which achieves the desired attributes.

\section{Attributes in Tension}
One of the defining characteristics of quality attributes is that they are often in conflict with each other.
It is a valiant yet wholly impractical pursuit to construct software which meets all quality attributes.

The role of a software architect is to identify which quality attributes are crucial to the success of their project,
and to implement practices to ensure the quality attribute is achieved.

The first law of software architecture \cite{richards2020fundamentals} states `Everything in software architecture is a trade-off'.
With the corollary of
`If an architect thinks they have discovered something that isn't a trade-off, more likely they just haven't identified the trade-off yet.'
