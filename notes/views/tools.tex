The important thing is that you should use a modelling tool, not a drawing tool.
Many drawing tools provide UML templates, and some also support C4.
The issue with drawing tools is that they do not know what the elements of the diagram mean.
If the name of an operation in a class is changed in a drawing tool, you will need to manually change it wherever it is referenced in other diagrams
(e.g. in sequence diagrams).
A modelling tool will track the information that describes the model, so that a change to a model element in one place,
will be replicated wherever that element appears in other diagrams.

There are many tools that support UML.
In a commercial project using UML on a large system, the cost of professional UML tools is negligible and is quickly recovered by the automation they provide.
There are a number of free UML tools. Some to consider are \link{Astah}{https://astah.net/products/free-student-license/},
\link{ModelIO}{https://www.modelio.org/}, or \link{PlantUML}{https://plantuml.com/}.
\link{Visual Paradigm}{https://www.visual-paradigm.com/} is not as recommended, as their free cloud-based tool is only a drawing tool, and not a modelling tool.

\begin{description}
    \item[Astah]
        is a commercial product that supports visual modelling in many notations. They provide a free UML tool for students.
    \item[ModelIO] is an open source visual UML modelling tool.
    \item[PlantUML] Is an open source text-based descriptive language that generates UML diagrams.
        \link{PlantText}{https://www.planttext.com/} is an online tool supporting it.
    \item[Visual Paradigm] is a commercial product that supports visual modelling in many notations.
        They provide a simple free cloud-based drawing tool that supports UML and some limited aspects of C4, but it lacks full modelling support.
\end{description}

\noindent
There are fewer tools that support C4. Some to consider are \link{Structurizr}{https://www.structurizr.com/},
\link{C4-PlantUML}{https://github.com/plantuml-stdlib/C4-PlantUML}, \link{Archi}{https://www.archimatetool.com/},
\link{IcePanel}{https://icepanel.io/}, or \link{Gaphor}{https://gaphor.org/}.

\begin{description}
    \item[Structurizr]
        was developed by Simon Brown as a tool to support generating C4 diagrams from textual descriptions.
        UQ students may register for free access to the paid version of the \link{Structurizr Cloud Service}{https://structurizr.com/help/academic}.
        You must use your \texttt{student.uq.edu.au} or \texttt{uq.net.au} email address when you register to get free access.
        Structurizr is an \link{open source tool}{https://github.com/structurizr/}.
        You can use a domain specific language to describe a C4 model, or you can embed the details in Java or .Net code.
    \item[C4-PlantUML] which extends PlantUML to support C4.
    \item[Archi] is an open source visual modelling tool that
        \link{supports C4}{https://www.archimatetool.com/blog/2020/04/18/c4-model-architecture-viewpoint-and-archi-4-7/} and ArchiMate models.
    \item[IcePanel] is a cloud-based visual modelling tool that supports C4. There is a limited free license for the tool.
    \item[Gaphor] is an open source visual modelling tool that supports UML and C4.
\end{description}

\subsection{Textual vs Visual Modelling}
The tools described above include both graphical and textual modelling tools.
Graphical tools, such as Astah, ModelIO, Archi and Gaphor, allow you to create models by drawing them.
This approach is often preferred by \link{visually oriented learners}{https://vark-learn.com/strategies/visual-strategies/}.
Text-based tools, such as PlantUML and Structurizr, allow you to create models by providing a textual description of the model.
This approach is often preferred by \link{read/write oriented learners}{https://vark-learn.com/strategies/readwrite-strategies/}.

Despite preferences, there are situations where there are advantages of using a text-based modelling tool.
Being text, the model can be stored and versioned in a version control system (e.g. git).
For team projects, it is much easier for everyone to edit the model and ensure that you do not destory other team members' work.
It is also possible to build a tool pipeline that will generate diagrams and embed them into the project documentation.

Text-based modelling tools, such as Structurizr or PlantUML, use a
\link{domain specific language}{https://opensource.com/article/20/2/domain-specific-languages} (DSL) to describe the model.
These tools require that you learn the syntax and semantics of the DSL.
The following sources of information will help you learn the C4 DSL:
\begin{itemize}[nosep]
    \item \link{language reference manual}{https://github.com/structurizr/dsl/blob/master/docs/language-reference.md},
    \item \link{language examples}{https://github.com/structurizr/dsl/tree/master/docs/cookbook}, 
    \item \link{on-line editable examples}{https://structurizr.com/dsl}, and
    \item \link{off-line tool}{https://github.com/structurizr/cli/blob/master/docs/getting-started.md}.
\end{itemize}
You may also find that the Sahara eCommerce C4 model is useful as an example of a number of features of the DSL.

The following sources of information will help you learn how to use the PlantUML DSL:
\begin{itemize}[nosep]
    \item \link{language reference manual}{https://plantuml.com/guide},
    \item \link{language overview with examples}{https://plantuml.com/}, and
    \item \link{on-line editable examples}{https://www.planttext.com/}.
\end{itemize}

\subsection{Example Diagrams}
You are able to download the UML and C4 models of the Sahara eCommerce example, from the course website.
The \link{UML model}{https://csse6400.uqcloud.net/resources/Sahara_eCommerce.asta}
was created using \link{Astah}{https://astah.net/products/free-student-license/}.
You may use this as an example of creating a model using a visual modelling tool.
There is a little more detail in the Astah model than what is shown in these notes.

The \link{C4 model}{https://csse6400.uqcloud.net/resources/c4_model.zip}
was created using the \link{Structurizr}{https://www.structurizr.com/} DSL
(\link{Domain Specific Language}{https://opensource.com/article/20/2/domain-specific-languages}).
You may use the C4 model as an example of creating a model using a textual description of the model (the DSL).
