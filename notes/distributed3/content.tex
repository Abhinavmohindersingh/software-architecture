\title{Distributed Computing III}
\author{Richard Thomas}
\date{\week{7}}

\maketitle

\section{Introduction}

In looking at distributed systems, we started from the perspective of Murphy's Law,
\textit{if anything can go wrong it will}.
We will now move on to O'Toole's Commentary, \textit{Murphy was an optimist}.

Large distributed systems consist of thousands of computing platforms,
communicating over large distances and over unreliable internet connections.
Failure of some part of the system is practically guaranteed \cite{datacenter-computer},
the system must be designed to cater for \emph{partial failure}.
Even for small systems, some part will eventually fail,
so fault handling must be part of the design.


\section{Consensus}

\subsection{Behaving Nodes}
Leaders \& Locks

\subsection{Byzantine Faults}
Byzantine Generals Problem

Idempotent

\section{Consistency}

\subsection{Eventual Consistency}

\subsection{Linearizability}

\subsection{CAP Theorem}
